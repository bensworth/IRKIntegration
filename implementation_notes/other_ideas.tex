\documentclass[a4paper,10pt]{article}
\usepackage[margin=1in]{geometry}
\usepackage{amsmath,amsthm,amssymb,amsfonts,stmaryrd}
\usepackage{soul}
\usepackage{array}
\usepackage{empheq}
\usepackage{xfrac}
\usepackage{minibox}
\usepackage{enumitem}
	\setlist{nosep} % or \setlist{noitemsep} to leave space around whole list
\usepackage{color}
\usepackage{blkarray}
\setcounter{MaxMatrixCols}{20}
\usepackage{showlabels}
\usepackage{arydshln}	% Dotted lines in arrays
\usepackage{adjustbox}
\usepackage{hyperref}
\hypersetup{
  colorlinks   = true, %Colours links instead of ugly boxes
  urlcolor     = blue, %Colour for external hyperlinks
  linkcolor    = blue, %Colour of internal links
  citecolor   = red %Colour of citations
}
\usepackage{cleveref}


\newtheorem{lemma}{Lemma}
\newtheorem{definition}{Definition}
\newtheorem{theorem}{Theorem}
\newtheorem{corollary}{Corollary}

\newcommand{\tcb}{\textcolor{blue}}
\newcommand{\todo}[1]{\textcolor{red}{[TODO\@: #1]}}


\begin{document}
\allowdisplaybreaks

% ---------------------------------------------------------------------------------------------- %
% ---------------------------------------------------------------------------------------------- %
% ---------------------------------------------------------------------------------------------- %
\section{The time-dependent case}

% ---------------------------------------------------------------------------------------------- %
% ---------------------------------------------------------------------------------------------- %
\subsection{LU/Gauss decomposition}

See Theorem 4.4 in ``A Theory of Noncommutative Determinants and Characteristic Functions of Graphs.''
This proves the standard LU/Gauss decomposition for a matrix over a noncommutative division ring.
It is based on quasideterminants. That paper gives a UL decomposition, but Theorem 4.9.7 in
arxiv version of QUASIDETERMINANTS gives the LU decomposition. In practice I think we would want
the LU, not UL, particularly for stiffly accurate because we don't need to compute the whole $U^{-1}$. 

Observe that the block matrix $\mathcal{M}_s$ in \eqref{eq:k1} is a matrix defined over a 
noncommutative ring $R$. Let us further assume that $R$ is a division ring, that is, every 
element in $R$ has a multiplicative inverse. Note that it is easy to construct an example
where $R$ is not a division ring: simply let $\alpha_{12}$ be an eigenvalue of $(\alpha_{11}I - \mathcal{L})$,
in which case $(\alpha_{11}I - \mathcal{L}) - \alpha_{12}I$ is singular and does not have
a multiplicative inverse. However, this assumption is a technical tool and \tcb{I believe
RK coefficients make it okay in practice?}

Moving forward, in matrix form a constant $\alpha_{ij}$ is typically implied to be
$\alpha_{ij}I$, but the $I$ is removed for ease of notation.


% ---------------------------------------------------------------------------------------------- %
\subsubsection{2x2}

\begin{align*}
L & = \begin{bmatrix}
	1 & 0 \\
	\alpha_{21} (\alpha_{11} - \mathcal{L}_1)^{-1} & I
\end{bmatrix}
,\hspace{5ex}
U = \begin{bmatrix}
	\alpha_{11} - \mathcal{L}_1 & \alpha_{12} \\
	0 & \alpha_{22} - \mathcal{L}_2 - \alpha_{12} \alpha_{21} (\alpha_{11} - \mathcal{L}_1)^{-1}
\end{bmatrix}
\end{align*}


% ---------------------------------------------------------------------------------------------- %
\subsubsection{3x3}

\begin{align*}
L & = 
\begin{bmatrix}
I & 0 & 0 \\
\alpha_{31} (\alpha_{11} - \mathcal{L}_1)^{-1} & I & 0 \\
\alpha_{21} (\alpha_{11} - \mathcal{L}_1)^{-1} & (\alpha_{22} - \mathcal{L}_2 - \alpha_{12} \alpha_{21} (\alpha_{11} - \mathcal{L}_1)^{-1}) (\alpha_{32} - 
\alpha_{12} \alpha_{31} (\alpha_{11} - \mathcal{L}_1)^{-1})^{-1} & I
\end{bmatrix} ,
\\
U & = \begin{bmatrix}
\alpha_{11} - \mathcal{L}_1 & \alpha_{12} & \alpha_{13} \\
0 & \alpha_{32} - \alpha_{12} \alpha_{31} (\alpha_{11} - \mathcal{L}_1)^{-1} & \alpha_{33} - \mathcal{L}_3 - \alpha_{13} \alpha_{31} (\alpha_{11} - \mathcal{L}_1)^{-1} \\
0 & 0 & Q
\end{bmatrix}
\\
Q & = \alpha_{23} - 
\alpha_{13} \alpha_{21} (\alpha_{11} - \mathcal{L}_1)^{-1} -
	(\alpha_{22} - \mathcal{L}_2 - \alpha_{12} \alpha_{21} (\alpha_{11} - \mathcal{L}_1)^{-1}) \\
	&\hspace{5ex}
	(\alpha_{33} - \mathcal{L}_3 - \alpha_{13} \alpha_{31} (\alpha_{11} - \mathcal{L}_1)^{-1})
	(\alpha_{32} - \alpha_{12} \alpha_{31} (\alpha_{11} - \mathcal{L}_1)^{-1})^{-1}
\\
& = \alpha_{23} - 
\alpha_{13} \alpha_{21} (\alpha_{11} - \mathcal{L}_1)^{-1} -
	(\alpha_{11} - \mathcal{L}_1)^{-1}((\alpha_{11} - \mathcal{L}_1)(\alpha_{22} - \mathcal{L}_2) -
		\alpha_{12} \alpha_{21}) \\
	&\hspace{5ex}
	((\alpha_{33} - \mathcal{L}_3)(\alpha_{11} - \mathcal{L}_1) - \alpha_{13} \alpha_{31})
	(\alpha_{32}(\alpha_{11} - \mathcal{L}_1) - \alpha_{12} \alpha_{31})^{-1}
\\
& = (\alpha_{11} - \mathcal{L}_1)^{-1} \bigg( \Big[\alpha_{23}(\alpha_{11} - \mathcal{L}_1) -
	\alpha_{13} \alpha_{21}\Big]\Big[\alpha_{32}(\alpha_{11} - \mathcal{L}_1) - \alpha_{12} \alpha_{31}\Big]
	- \\\
	&\hspace{5ex}
	\Big[(\alpha_{11} - \mathcal{L}_1)(\alpha_{22} - \mathcal{L}_2) - \alpha_{12} \alpha_{21}\Big]
	\Big[(\alpha_{33} - \mathcal{L}_3)(\alpha_{11} - \mathcal{L}_1) - \alpha_{13} \alpha_{31}\Big]
	\bigg)\Big[\alpha_{32}(\alpha_{11} - \mathcal{L}_1) - \alpha_{12} \alpha_{31}\Big]^{-1}
\end{align*}



Quasideterminant:
{\tiny
\begin{align*}
\begin{bmatrix}
\alpha_{33} - \mathcal{L}_3 - 
 \alpha_{32} \left(\alpha_{13} \left(\alpha_{12} - \left(\alpha_{11} - \mathcal{L}_1\right) \frac{\left(\alpha_{22} - \mathcal{L}_2\right)}{\alpha_{21}}\right)^{-1} - \frac{\left(
    \alpha_{23} \left(\alpha_{12} - \left(\alpha_{11} - \mathcal{L}_1\right) \frac{\left(\alpha_{22} - \mathcal{L}_2\right)}{\alpha_{21}}\right)^{-1} \left(\alpha_{11} - \mathcal{L}_1\right)\right)}{\alpha_{21}}\right) - 
 \alpha_{31} \left(-\left(\frac{\left(\alpha_{13} \left(\alpha_{22} - \mathcal{L}_2\right) \left(\alpha_{12} - \left(\alpha_{11} - \mathcal{L}_1\right) \left(\alpha_{22} - \mathcal{L}_2\right)/\alpha_{21}\right)^{-1}\right)}{\alpha_{21}}\right)
 + \frac{\left(\alpha_{23} \left(1 + \left(\alpha_{22} - \mathcal{L}_2\right) \left(\alpha_{12} - \left(\alpha_{11} - \mathcal{L}_1\right) \frac{\left(\alpha_{22} - \mathcal{L}_2\right)}{
          \alpha_{21}}\right)^{-1} \left(\alpha_{11} - \mathcal{L}_1\right)/\alpha_{21}\right)\right)}{\alpha_{21}}\right)
\end{bmatrix}
\end{align*}
}


% ---------------------------------------------------------------------------------------------- %
% ---------------------------------------------------------------------------------------------- %
\subsection{Direct inverse?}

Consider 
%
\begin{align}\label{eq:Mnt}
\mathcal{M}_3 &:= \begin{bmatrix} (\alpha_{11} I + \mathcal{L}) & \alpha_{12}I & \alpha_{13}I \\
    \alpha_{21}I & (\alpha_{22}I + \mathcal{L}) & \alpha_{23} I \\
    \alpha_{31}I & \alpha_{32}I & (\alpha_{33} I + \mathcal{L}) \end{bmatrix}.
\end{align}
%
For stiffly accurate problems, we only need the last row of $\mathcal{M}_3^{-1}$. Using
the Noncommutative algebra package in mathematica, we can work this out as
%
\begin{align*}
P & := \frac{1}{\alpha_{21}} \bigg(\alpha_{13}\alpha_{21} - \alpha_{23} (\alpha_{11} - \mathcal{L}_1)
 	+ \Big[\alpha_{12}\alpha_{21} - (\alpha_{11} - \mathcal{L}_1)(\alpha_{22} - \mathcal{L}_2)\Big]
 	\\ &\hspace{20ex} 
 	\Big[\alpha_{32}\alpha_{21} - \alpha_{31} (\alpha_{22} - \mathcal{L}_2)\Big]^{-1}
 	\Big[\alpha_{23}\alpha_{31} - \alpha_{21}(\alpha_{33} - \mathcal{L}_3)\Big]\bigg),
\\
r_1 & = P^{-1} \\
r_2 & = -\frac{1}{\alpha_{21}} P^{-1} \left( (\alpha_{11} - \mathcal{L}_1) -
\alpha_{31} \Big[\alpha_{12}\alpha_{21} - (\alpha_{11} - \mathcal{L}_1)
	(\alpha_{22} - \mathcal{L}_2)\Big]
	\Big[\alpha_{32}\alpha_{21} - \alpha_{31} (\alpha_{22} - \mathcal{L}_2)\Big]^{-1}\right),
\\
r_3 & = - P^{-1} \Big[\alpha_{12}\alpha_{21} - (\alpha_{11} - \mathcal{L}_1) (\alpha_{22} - \mathcal{L}_2)\Big] \Big[\alpha_{32}\alpha_{21} -
\alpha_{31}(\alpha_{22} - \mathcal{L}_2)\Big]^{-1}
\end{align*}
%
For purposes of preconditioning, let us assume commutation of $\mathcal{L}_1$ and
$\mathcal{L}_2$ or $\mathcal{L}_2$ and $\mathcal{L}_3$ and factor out the inverse
term in $P$. This results in the modified operator
%
\begin{align*}
\widehat{P} &:= \frac{1}{\alpha_{21}} \bigg(
	\Big[\alpha_{13}\alpha_{21} - \alpha_{23} (\alpha_{11} - \mathcal{L}_1)\Big]
	\Big[\alpha_{32}\alpha_{21} - \alpha_{31} (\alpha_{22} - \mathcal{L}_2)\Big] +
\\ &\hspace{10ex} 
 	\Big[\alpha_{12}\alpha_{21} - (\alpha_{11} - \mathcal{L}_1)(\alpha_{22} - \mathcal{L}_2)\Big]
 	\Big[\alpha_{23}\alpha_{31} - \alpha_{21}(\alpha_{33} - \mathcal{L}_3)\Big]\bigg)
\end{align*}
%


{\color{blue}
Interestingly, I think $\widehat{P}$ is exactly the determinant in \eqref{eq:det}.. Upside
of that is maybe using the determinant/adjugate form would avoid having to apply the action
of $P$ (which includes an inverse) every iteration. Downside is it seems like that doesn't
take advantage of the fact that we only need $[0,...,0,1]\mathcal{M}_s^{-1}$.
}


% ---------------------------------------------------------------------------------------------- %
% ---------------------------------------------------------------------------------------------- %
\subsection{Two stages}

Consider the case of two fully implicit stages. Then we need to invert a block linear system
$\mathcal{M}_2\mathbf{s} = \mathbf{r}$, given by
%
\begin{align}\label{eq:Mnt}
\begin{bmatrix} \alpha_{11} I - \mathcal{L}_1 & \alpha_{12}I \\ \alpha_{21}I & \alpha_{22}I - \mathcal{L}_2\end{bmatrix}
	\begin{bmatrix}\mathbf{s}_1 \\ \mathbf{s}_2 \end{bmatrix} = 
	\begin{bmatrix}\mathbf{r}_1 \\ \mathbf{r}_2 \end{bmatrix}.
\end{align}
%
In the context of $2\times 2$ block operators, the key to inverting the matrix is inverting one of
the Schur complements. Define the matrix polynomials
%
\begin{align*}
P_{\mathcal{L}} := (\alpha_{11}I - \mathcal{L}_1)(\alpha_{22}I - \mathcal{L}_2) - \alpha_{12}\alpha_{21}I, \\
Q_{\mathcal{L}} := (\alpha_{22}I - \mathcal{L}_2)(\alpha_{11}I - \mathcal{L}_1) - \alpha_{12}\alpha_{21}I,
\end{align*}
%
and consider both Schur complements,
%
\begin{align*}
S_{22} & = \alpha_{22} I - \mathcal{L}_2 - \alpha_{12}\alpha_{21}(\alpha_{11} I - \mathcal{L} )^{-1} \\
& = \left[ ( \alpha_{22} I - \mathcal{L}_2)(\alpha_{11} I - \mathcal{L}_1) - \alpha_{12}\alpha_{21}I \right]
	(\alpha_{11} I - \mathcal{L}_1 )^{-1} \\
& = Q_{\mathcal{L}}(\alpha_{11} I - \mathcal{L}_1 )^{-1} \\
& = (\alpha_{11} I - \mathcal{L}_1 )^{-1}P_{\mathcal{L}} \\
S_{11} & = P_{\mathcal{L}}(\alpha_{22} I - \mathcal{L}_2 )^{-1} \\
& = (\alpha_{22} I - \mathcal{L}_2 )^{-1}Q_{\mathcal{L}}.
\end{align*}
%
%Note that 
%%
%\begin{align*}
%P_{\mathcal{L}} & = (\alpha_{22} I + \mathcal{L}_2 )Q_{\mathcal{L}}(\alpha_{22} I + \mathcal{L}_2 )^{-1} \\
%& = (\alpha_{11} I + \mathcal{L}_1 )^{-1} Q_{\mathcal{L}}(\alpha_{11} I + \mathcal{L}_1 ).
%\end{align*}
%%
Appealing to the closed form inverse of $2\times 2$ block matrices derived from a block LDU decomposition,
we can then write $\mathcal{M}_2^{-1}$ in terms of the Schur complements:
%
\begin{align}
\begin{bmatrix} \alpha_{11} I - \mathcal{L}_1 & \alpha_{12}I \\ \alpha_{21}I & \alpha_{22}I - \mathcal{L}_2\end{bmatrix} ^{-1}
	& = \begin{bmatrix}  (\alpha_{22} I - \mathcal{L}_2 )P_{\mathcal{L}}^{-1} & -\alpha_{12}Q_{\mathcal{L}}^{-1} \\
		-\alpha_{21}P_{\mathcal{L}}^{-1} & (\alpha_{11} I - \mathcal{L}_1 )Q_{\mathcal{L}}^{-1} \end{bmatrix} \nonumber\\
& = \begin{bmatrix} \alpha_{22} I - \mathcal{L}_2  & -\alpha_{12}I \\ -\alpha_{21}I & \alpha_{11} I - \mathcal{L}_1 \end{bmatrix} 
	\begin{bmatrix} P_{\mathcal{L}}^{-1}  & \mathbf{0} \\ \mathbf{0} & Q_{\mathcal{L}}^{-1} \end{bmatrix}.\label{eq:Minv}
\end{align}
%
\tcb{Note, this more or less exactly takes the form of a scalar $2\times 2$ matrix inverse, replacing
the $1/det$ with the right scaling by the polynomial inverses.. We can also restructure to pull
polynomials out of the left }


% ---------------------------------------------------------------------------------------------- %
\subsubsection{Two stages}

Returning to the case of two stages, by assuming $\mathcal{L}_1 = \mathcal{L}_2$ we have
$P_{\mathcal{L}} = Q_{\mathcal{L}} := P_2(\mathcal{L})$. Then, $\mathcal{M}_2^{-1}$ \eqref{eq:Minv} can
be applied through two applications of $P_2(\mathcal{L})^{-1}$, along with some additional mat-vecs and
vector addition. Note that $P_2(\mathcal{L})$ is a quadratic polynomial in $\mathcal{L}$, which can be
solved in two steps by factoring \eqref{eq:fac} and applying the successive inverses
$(\lambda_1 I - \mathcal{L})^{-1}(\lambda_2 I - \mathcal{L})^{-1}$, where $\lambda_1,\lambda_2$ are
the eigenvalues of $\boldsymbol{\alpha} = A_0^{-1}$. Note, in the $2\times 2$ case these have a
relatively simple algebraic structure, 
%
\begin{align*}
\lambda_1,\lambda_2 & = \frac{1}{2} \left( \alpha_{11} + \alpha_{22} \pm \sqrt{ (\alpha_{11} + \alpha_{22})^2 - 4 ( \alpha_{11}\alpha_{22} - \alpha_{12}\alpha_{21})} \right).
\end{align*}
%



% ---------------------------------------------------------------------------------------------- %
\subsubsection{Three stages}

For nonsingular scalar $3\times 3$ matrix $A$, the inverse is given by 
%
\begin{align}
{A}^{-1} =
\begin{bmatrix}
	a_{11} & a_{12} & a_{13}\\ a_{21} & a_{22} & a_{23} \\ a_{31} & a_{32} & a_{33}\\
\end{bmatrix}^{-1} =
\det({A})^{-1} \begin{bmatrix}
	\, A_{11} & \, A_{12} & \,A_{13} \\ \, A_{21} & \, A_{22} & \,A_{23} \\ \, A_{31} & \,A_{32} & \, A_{33}\\
\end{bmatrix},\label{eq:3inv}
\end{align}
%
with elements defined by
%
\begin{alignat*}{6}
A_{11} &={}&  (a_{22}a_{33} - a_{23}a_{32}), &\quad&
    A_{12} &={}& -(a_{12}a_{33} - a_{13}a_{32}), &\quad&
    A_{13} &={}&  (a_{12}a_{23} - a_{13}a_{22}), \\
A_{21} &={}& -(a_{21}a_{33} - a_{23}a_{31}), &\quad&
    A_{22} &={}&  (a_{11}a_{33} - a_{13}a_{31}), &\quad&
    A_{23} &={}& -(a_{11}a_{23} - a_{13}a_{21}), \\
A_{31} &={}&  (a_{21}a_{32} - a_{22}a_{31}), &\quad&
    A_{32} &={}& -(a_{11}a_{32} - a_{12}a_{31}), &\quad&
    A_{33} &={}&  (a_{11}a_{22} - a_{12}a_{21}),
\end{alignat*}
%
and determinant $\det({A}) = a_{11}A_{11}+a_{12}A_{21}+a_{13}A_{31}$.

In our case, consider 
%
\begin{align}\label{eq:Mnt}
\mathcal{M}_3 &:= \begin{bmatrix} (\alpha_{11} I + \mathcal{L}) & \alpha_{12}I & \alpha_{13}I \\
    \alpha_{21}I & (\alpha_{22}I + \mathcal{L}) & \alpha_{23} I \\
    \alpha_{31}I & \alpha_{32}I & (\alpha_{33} I + \mathcal{L}) \end{bmatrix}.
\end{align}
%
Define $\mathcal{N}_3$ as a block $3\times 3$ matrix with entries of $A^{-1}$ as in \eqref{eq:3inv},
excluding the determinant. Plugging in, we have entries of $\mathcal{N}_3$ given by
%
\begin{align*}
A_{11} &=  (\alpha_{22}I - \mathcal{L})(\alpha_{33} I - \mathcal{L}) - \alpha_{23}\alpha_{32}I, \\
    A_{12} &= -\alpha_{12}(\alpha_{33} I - \mathcal{L}) + \alpha_{13}\alpha_{32}I, \\
    A_{13} &=  \alpha_{12}\alpha_{23} I - \alpha_{13}(\alpha_{22}I - \mathcal{L}), \\
A_{21} &= -\alpha_{21}(\alpha_{33} I - \mathcal{L}) + \alpha_{23} \alpha_{31}I, \\
    A_{22} &=  (\alpha_{11} I - \mathcal{L})(\alpha_{33} I - \mathcal{L}) - \alpha_{13}\alpha_{31}I, \\
    A_{23} &= -\alpha_{23}(\alpha_{11} I - \mathcal{L}) + \alpha_{13}\alpha_{21}I, \\
A_{31} &=  \alpha_{21}\alpha_{32}I - \alpha_{31}(\alpha_{22}I - \mathcal{L}), \\
    A_{32} &= -\alpha_{32}(\alpha_{11} I - \mathcal{L}) + \alpha_{12}\alpha_{31}I, \\
    A_{33} &=  (\alpha_{11} I - \mathcal{L})(\alpha_{22}I - \mathcal{L}) - \alpha_{12}\alpha_{21}I.
\end{align*}
%
Working through the details, it is striaghtforward to confirm that $\mathcal{N}_3\mathcal{M}_3$
is a block-diagonal matrix, with diagonal blocks given by the (block) determinant of $\mathcal{M}_3$,
%
\begin{align} \label{eq:det}
\begin{split}
D & = (\alpha_{11}I - \mathcal{L}) (\alpha_{22}I - \mathcal{L})(\alpha_{33} I - \mathcal{L}) -
	\alpha_{23}\alpha_{32}(\alpha_{11}I - \mathcal{L}) - \\
& \hspace{5ex} 
	\alpha_{13}\alpha_{31}(\alpha_{22}I - \mathcal{L}) -
	\alpha_{12}\alpha_{21}(\alpha_{33}I - \mathcal{L}) + (\alpha_{13}\alpha_{32}\alpha_{21} + \alpha_{12}\alpha_{23}\alpha_{31})I.
\end{split}
\end{align}
%
Similar to the $2\times 2$ case (albeit algebraically more complicated), this is a cubic polynomial in
$\mathcal{L}$. By computing the roots of this polynomial, we can construct error propagation of a
three-stage fixed-point iteration that produces the exact inverse of $\mathcal{D}$. 


%
{\color{blue}A few more thoughts:
\begin{itemize}
\item Working through the algebra, the cancellation does not fully happen if $\mathcal{L}_i$
is time-dependent. However, a lot of it does. There will be some off-diagonal terms that take
the form, for example,
%
\begin{align*}
(\alpha_{11}I - \mathcal{L}_1)(\alpha_{22}I - \mathcal{L}_2) - 
	(\alpha_{22}I - \mathcal{L}_2)(\alpha_{11}I - \mathcal{L}_1)
& = \mathcal{L}_1\mathcal{L}_2 - \mathcal{L}_2\mathcal{L}_1.
\end{align*}
%
This provides a nice theoretical tool to analyze what is going on. It is possible these are often
quite small. Moreover, we may be able to choose an ordering where these terms only occur on,
say, the stritly lower triangular part, in which case a block-triangular preconditioning would also
be exact. 

\end{itemize}
}
%

% ---------------------------------------------------------------------------------------------- %
% ---------------------------------------------------------------------------------------------- %
% ---------------------------------------------------------------------------------------------- %
\section{Parallel in time}

\begin{itemize}
	\item For pAIR or MGRiT, can we do a $p$-multigrid approach, where we solve order $p$ MGRIT
	via a set of $p$ MGRiT/pAIR solves for order 1? 
	\item Another (simpler) possibility is use global Krylov only applied to the solution vector,
	and precondition a high-order time integrator with pAIR built on a low-order time integrator.
	This could also be done as a two-grid method, where we do high-order stage integration as a
	relaxation coupled with a pAIR solve with a low-order time integrator. This could couple well
	if I come up with a good way to do fully implicit IRK solves, we could use that on the finest
	grid. Or, we could just use full implicit IRK as the operator, relax w/ high order SDIRK,
	then coarse-grid correct with pAIR on low-order SDIRK (is this SDC or PFAST?)
\end{itemize}


% ---------------------------------------------------------------------------------------------- %
% ---------------------------------------------------------------------------------------------- %
\subsection{Precondition HO with LO}


Let $\Phi$ denote a single step of our high-order (fine-grid) Runge-Kutta operator and
$\Psi$ a single step of our low-order (coarse-grid) operator. Then error propagation for
preconditioning a global time-stepping scheme base don $\Phi$ with a low-order scheme 
based on $\Psi$ is given by 
%
\begin{align*}
I - B_\Delta^{-1} A & =
	{\begin{bmatrix} \mathbf{0} \\ I & \mathbf{0} \\ \Psi & I & \mathbf{0} \\ \vdots & \vdots & \ddots & \ddots \\ \Psi^{N_c-2} & \Psi^{N_c-3} & ... & I  & \mathbf{0} \end{bmatrix} } \textnormal{{diag}}{(\Psi - \Phi)}.
\end{align*}
%
Here we have assumed $\Phi$ and $\Psi$ to be linear and independent of time, but that
is not necessary. The (linearized) time-dependent setting has the same form as above,
but with products of $\Psi$ evaluated at successive (discrete) times.

Note this is analogous to two-level convergence of MGRiT. Necessary and sufficient conditions
for convergence are given by the temporal approximation property,
%
\begin{definition}[Temporal approximation property]
Let $\Phi$ denote a fine-grid time-stepping operator and $\Psi$ denote a coarse-grid time-stepping operator, for all time points,
with coarsening factor $k$. Then, $\Phi$ satisfies an F-relaxation temporal approximation property with power $p$
(F-TAP$_p$), with respect to $\Psi$, with constant $\varphi_{F,p}$, if, for all vectors $\mathbf{v}$,
\begin{align}\label{eq:tap_f}
\|(\Psi - \Phi^k)^p\mathbf{v}\| \leq \varphi_{F,p} \left[\min_{x\in[0,2\pi]} \left\| (I - e^{\mathrm{i}x}\Psi )^p\mathbf{v}\right\| \right].
\end{align}
Similarly, $\Phi$ satisfies an FCF-relaxation temporal approximation property with power $p$ (FCF-TAP$_p$), with
respect to $\Psi$, with constant $\varphi_{FCF,p}$, if, for all vectors $\mathbf{v}$,
\begin{align}\label{eq:tap_fcf}
\|(\Psi - \Phi^k)^p\mathbf{v}\| \leq \varphi_{FCF,p}\left[\min_{x\in[0,2\pi]} \left\| (\Phi^{-k}(I - e^{\mathrm{i}x}\Psi) )^p\mathbf{v}\right\| \right].
\end{align}
\end{definition}
%
For the F-relaxation case, we have $k = 1$. The analogous FCF context here would presumably
correspond to doing a complete fine-grid solve on processor. The downside is this is kind of
expensive. However, if we treat $\Phi$ as an explicit scheme, with the stages and implicit 
solves included, just the action of the operator will be expensive to compute. I think it will
be important to somehow consider the RK scheme expanded, so that when we apply a coarse-grid
preconditioner, computing the action of the operator can be done explicitly and doesn't require
computing a bunch of inverses...

L-stability appears to be important here, and is not perfect. But, with just F-relaxation, then
we don't even need to solve the fine grid implicit discretization. Need to check how this could
work with fully implicit fine grid discretization, preconditioned with BE.. Would be cool if that
could be effective. 


% ---------------------------------------------------------------------------------------------- %
% ---------------------------------------------------------------------------------------------- %
\subsection{A multigrid or block preconditioning perspective}

Let $A$ and $\mathbf{b}$ correspond to Butcher tableaux,
%
\begin{align*}
\hspace{-6ex}\implies \hspace{3ex}\mathbf{u}_{i+1} & = \mathbf{u}_i + \delta t\sum_{i=1}^{s} b_i\mathbf{k}_i,
\end{align*}
%
and let $\hat{\mathbf{u}} = [\mathbf{u}_0, \mathbf{u}_1,...]$ denote the space-time solution
at times $t_0,t_1,...$. Then the full space-time linear system takes the form $A\hat{\mathbf{u}}
= \hat{\mathbf{g}}$, where
%
\begin{align*}
\hspace{-4ex}A = 
\begin{blockarray}{c c c c c c} \textnormal{\underline{\tiny{$t_0$}}} & \textnormal{\tiny(stages)} &
	\textnormal{\underline{\tiny{$t_1$}}} &  \textnormal{\tiny(stages)} & \textnormal{\underline{\tiny{$t_2$}}} & ... \\
\begin{block}{(c c c c c c)}
	I &  \\
	-\delta t(\mathbf{1}_n \otimes \mathcal{L}) & I - \delta t A\otimes \mathcal{L} &\\
	-I & -\mathbf{b}\otimes I & I \\ 
	& & -\delta t(\mathbf{1}_n \otimes \mathcal{L}) & I - \delta t A\otimes \mathcal{L} &  \\
	& & -I & -\mathbf{b}\otimes I & I \\
	&&& \ddots\text{\hspace{4ex}}  & \ddots\text{\hspace{8ex}}  & \ddots \\
	\end{block}
\end{blockarray}\hspace{1ex}.
\end{align*}
%
Note, for ease of notation here, we have assumed that $\mathcal{L}$ is independent of time
and the stage matrix can be written as a Kronecker product. However, for time-dependent
problems, the space-time matrix takes a similar structure, but the stage-matrix must be
written explicitly as a block $s\times s$ matrix. 
Now, let us denote all intermediate RK stages as F-points and time-points as C-points. 
Eliminating the F-points in a Schur complement sense yields the reduced MGRiT-style
system,
%
\begin{align*}
A & = \begin{blockarray}{c c c c c} \textnormal{\underline{\tiny{$t_0$}}} & \textnormal{\underline{\tiny{$t_1$}}} & ... &... &\textnormal{\underline{\tiny{$t_N$}}} \\
	\begin{block}{(c c c c c)}
	I & & & & \\
	-\Phi & I & & &\\
	 & -\Phi& I & & \\
	 &  & \ddots & \ddots  & \\
	 &  & & -\Phi & I \\
	\end{block}
	\end{blockarray} \hspace{1ex}, 
\end{align*}
%
where
%
\begin{align}\label{eq:mgrit}
\Phi & = I + \delta t\mathbf{b}_0^T\otimes I \left(I - \delta t A_0\otimes \mathcal{L}\right)^{-1} (\mathbf{1}_n \otimes \mathcal{L}).
\end{align}
%
As previously, this takes a more complicated but similar form if $\mathcal{L}$ is
time-dependent.\\
\\
I see two nice ways to think about this:
%
\begin{enumerate}
	\item The simplest is a block-preconditioning approach. Reordering the matrix by stages
	and then time-points, We will use a block lower-triangular preconditioner. If we exactly
	invert the F-block (solve all RK stages exactly) followed by inverting the Schur complement,
	we converge in two iterations. Of course in this case, that is just sequential time stepping.
	But, we could approximate our Schur complement with a space-time solve using a low-order 
	integrator (BE?) and couple this with an F-relaxation (some inexact solution to the
	stage matrices). Note, theory and practice have generally indicated that if one solve is
	not exact, the other also does not need to be, so we probably shouldn't solve the stages
	exactly. This makes me think we might be able to let the true RK scheme be a high-order
	fully implicit RK scheme, and precondition this with an SDIRK scheme on stage integration.

	\item Above can also be thought of in some sense as a non-Galerkin multigrid method.
	But, we developed a decent two-grid theoretical framework for AIR. I wonder if we can
	construct $R$ and $P$ so that approximating high-order $\Phi$ on the coarse-grid with
	backward Euler is actually a Petrov-Galerkin multigrid method, that is, $RAP$ directly
	leads to a matrix analogous to \eqref{eq:mgrit} but with $\Phi\sim$ backward Euler
	(or some other one-stage method)?

\end{enumerate}





\end{document}

