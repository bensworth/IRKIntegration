\documentclass[a4paper,10pt]{article}
\usepackage[margin=1in]{geometry}
\usepackage{amsmath,amsthm,amssymb,amsfonts,stmaryrd}
\usepackage{soul}
\usepackage{array}
\usepackage{empheq}
\usepackage{xfrac}
\usepackage{minibox}
\usepackage{enumitem}
	\setlist{nosep} % or \setlist{noitemsep} to leave space around whole list
\usepackage{color}
\usepackage{blkarray}
\setcounter{MaxMatrixCols}{20}
\usepackage{showlabels}
\usepackage{arydshln}	% Dotted lines in arrays
\usepackage{adjustbox}
\usepackage{hyperref}
\hypersetup{
  colorlinks   = true, %Colours links instead of ugly boxes
  urlcolor     = blue, %Colour for external hyperlinks
  linkcolor    = blue, %Colour of internal links
  citecolor   = red %Colour of citations
}
\usepackage{cleveref}


\newtheorem{lemma}{Lemma}
\newtheorem{definition}{Definition}
\newtheorem{theorem}{Theorem}
\newtheorem{corollary}{Corollary}

\newcommand{\tcb}{\textcolor{blue}}
\newcommand{\todo}[1]{\textcolor{red}{[TODO\@: #1]}}


\begin{document}
\allowdisplaybreaks

% ---------------------------------------------------------------------------------------------- %
% ---------------------------------------------------------------------------------------------- %
% ---------------------------------------------------------------------------------------------- %
\section{Introduction}

Consider the method-of-lines approach to solving partial differential equations (PDEs),
where we discretize in space and arrive at a system of DAEs in time,
%
\begin{align}\label{eq:problem}
\begin{split}
	M\mathbf{u}'(t) & =  \mathcal{N}(\mathbf{u},p,t), \\ 
	{0} & = \mathcal{G}(\mathbf{u}),
\end{split}
\end{align}
%
for $t\in(0,T]$ and initial conditions $\mathbf{u}(0) = \mathbf{u}_0, p= p_0$.
Here $M$ is a mass matrix, $\mathcal{N}\in\mathbb{R}^{N\times N}$ a discrete,
time-dependent, (non)linear operator depending on variables $t$, $\mathbf{u}$,
and $p$ (including potential forcing terms), and $\mathcal{G}(\mathbf{u})$ is
an algebraic constraint. We assume the DAE is index-2, where it cannot be
immediately eliminated to arrive at an ODE, but for which we can apply the
theory in Hairer on RK integration. Note, this encompasses many classical PDE
problems, in particular, the divergence-free constraint that arises in fluid
and plasma flow problems.

Now consider time propagation using an $s$-stage Runge-Kutta scheme,
characterized by the Butcher tableaux 
%
\begin{align*}
	\renewcommand\arraystretch{1.2}
	\begin{array}
	{c|c}
	\mathbf{c}_0 & A_0\\
	\hline
	& \mathbf{b}_0^T
	\end{array},
\end{align*}
%
with Runge-Kutta matrix $A_0 = (a_{ij})$, weight vector $\mathbf{b}_0^T =
(b_1, \ldots, b_s)^T$, and nodes $\mathbf{c}_0 = (c_0, \ldots, c_s)$. We use
the $\varepsilon$-embedding approacg to Runge-Kutta integration, where one can
treat \eqref{eq:problem} as the singular limit of a problem with time
derivative $\varepsilon p'$ as $\varepsilon \to 0$. The resulting method
takes the form
%
\begin{align*}
\mathbf{u}_{n+1} & = \mathbf{u}_n + \delta t \sum_{i=1} b_i\mathbf{k}_i, \\
p_{n+1} & = p_n + \delta t \sum_{i=1}^s b_i{\ell}_i,
\end{align*}
%
where 
%
\begin{align}\label{eq:UP}
U_i & = \mathbf{u}_n + \delta t\sum_{j=1}^s a_{ij} \mathbf{k}_j, \hspace{5ex}
P_i = p_n + \delta t\sum_{j=1}^s a_{ij} \ell_j,
\end{align}
%
which satisfy the equations
%
\begin{align}\label{eq:kl}
\mathbf{k}_i & = \mathcal{N}(U_i,P_i, t_i), \hspace{5ex} \mathcal{G}(U_i) = 0.
\end{align}
%

This appears a bit cumbersome, so let assume for a given time step we have
linearized using, for example, a Newton or a Picard linearization of the
underlying PDE. Then we can pose \eqref{eq:problem} in linear block form
via
%
\begin{align}\label{eq:problem2}
\begin{split}
\begin{bmatrix} M \mathbf{u}_t \\ 0 \end{bmatrix}
	= \begin{bmatrix} L_{uu} & L_{up} \\ L_{pu} & S \end{bmatrix}
	\begin{bmatrix} \mathbf{u} \\ p \end{bmatrix} + 
	\begin{bmatrix} g \\ 0 \end{bmatrix}.
\end{split},
\end{align}
%
where $S$ (potentially $= 0$) is a stabilization matrix.
In principle, I believe the forcing vectors can be nonzero in the constraint,
but in focusing on flow-problems with a divergence-free
constraint, we will assume the forcing function on $p$ is zero. We can
now expand \eqref{eq:UP} and \eqref{eq:kl} to express the stage vectors
$\{\mathbf{k}_j, \ell_j\}$ as the solution of a large block linear system,
%
\begin{align}\label{eq:k0}
\left( \begin{bmatrix} \begin{bmatrix} M \\ & \mathbf{0}\end{bmatrix} & &
	\mathbf{0} \\ & \ddots \\ \mathbf{0} & & \begin{bmatrix} M \\ & \mathbf{0}\end{bmatrix}
		\end{bmatrix}
	- \delta t \begin{bmatrix} a_{11}\begin{bmatrix} L^{(1)}_{uu} & L^{(1)}_{up} \\ L^{(1)}_{pu} & S \end{bmatrix} & ... & a_{1s}\begin{bmatrix} L^{(1)}_{uu} & L^{(1)}_{up} \\ L^{(1)}_{pu} & S \end{bmatrix} \\
	\vdots & \ddots & \vdots \\ a_{s1}\begin{bmatrix} L^{(s)}_{uu} & L^{(s)}_{up} \\ L^{(s)}_{pu} & S \end{bmatrix} & ... & a_{ss} \begin{bmatrix} L^{(s)}_{uu} & L^{(s)}_{up} \\ L^{(s)}_{pu} & S \end{bmatrix} \end{bmatrix} \right)
	\begin{bmatrix} \mathbf{k}_1 \\ \ell_1 \\ \vdots \\ \mathbf{k}_s \\ \ell_s\end{bmatrix} 
& = \begin{bmatrix} \mathbf{f}_1 \\ \mathbf{0} \\ \vdots \\ \mathbf{f}_s \\ \mathbf{0} \end{bmatrix},
\end{align}
%
where $\mathbf{f}_i = g(t_i) + L_{uu}\mathbf{u} ...$ and superscripts on
spatial operators $L$ denote the time at which the spatial operator is
evaluated (e.g., as a Jacobian in Newton). \tcb{Address right-hand side
vectors correctly and fix in all following equations.}

Note, the first term in \eqref{eq:k0} can be expressed as
$I_s \otimes \begin{bmatrix} M \\ & \mathbf{0}\end{bmatrix}$, for $s\times s$
identity matrix $I_s$. Now, let us insert between the matrix and vector in
\eqref{eq:k0} the identity $(A_0^{-1}\otimes I)(A_0\otimes I)$. Distributing the
$A_0^{-1}$ to the matrix yields the equivalent system
%
\begin{align}\label{eq:k1}
\left(A_0^{-1} \otimes \begin{bmatrix} M \\ & \mathbf{0}\end{bmatrix}
	- \delta t \begin{bmatrix} \begin{bmatrix} L^{(1)}_{uu} & L^{(1)}_{up} \\ L^{(1)}_{pu} & S \end{bmatrix} & \\
	\ & \ddots & \ \\ & & \begin{bmatrix} L^{(s)}_{uu} & L^{(s)}_{up} \\ L^{(s)}_{pu} & S \end{bmatrix} \end{bmatrix} \right)
	(A_0\otimes I)\begin{bmatrix} \mathbf{k}_1 \\ \ell_1 \\ \vdots \\ \mathbf{k}_s \\ \ell_s\end{bmatrix} 
& = \begin{bmatrix} \mathbf{f}_1 \\ \mathbf{0} \\ \vdots \\ \mathbf{f}_s \\ \mathbf{0} \end{bmatrix},
\end{align}
%

% ---------------------------------------------------------------------------------------------- %
% ---------------------------------------------------------------------------------------------- %
% ---------------------------------------------------------------------------------------------- %
\section{Index-1 DAEs}

\begin{itemize}
	\item For index-1 DAEs, can eliminate constraint and yield system of ODEs. 
	\item Under reasonable assumptions on stability of resulting system of ODEs,
	same theory from linear paper can be applied to the reduced system.
	\item Try to show that directly applying IRK block preconditioning to reduced
	ODE system equivalent to applying similar preconditioning to DAE form (think
	this should hold). Then, algorithm seamlessly extends to index-1 DAEs.
	\item Use KHI model as example problem in paper.

	\item For nonlinear constraint, what is relation between elimination of
	linearized constraint in block preconditioning form and elimination of
	nonlinear constraint to form reduced ODE system? 

\end{itemize}




% ---------------------------------------------------------------------------------------------- %
% ---------------------------------------------------------------------------------------------- %
% ---------------------------------------------------------------------------------------------- %
\section{Linear/simplified Newton}

Suppose we are solving a linear problem, or a nonlinear problem with a simplified
Newton method, where the Jacobian is evaluated at a single time point for all
stages. Let $\mathbf{K}$ denote the stage vector and $\mathbf{F}$ the right-hand
side. Then, \eqref{eq:k0} can be expressed in a further condensed Kronecker
product form as
%
\begin{align}\label{eq:k2}
\left(A_0^{-1} \otimes \begin{bmatrix} M \\ & \mathbf{0}\end{bmatrix}
	- \delta t I_s \otimes \begin{bmatrix} L_{uu} & L_{up} \\ L_{pu} & S \end{bmatrix} \right)
	(A_0\otimes I)\mathbf{K} & = \mathbf{F}.
\end{align}
%
We can simplify this further by appealing to the real Schur decomposition,
$A_0^{-1} = Q_0R_0Q_0^T$, where $Q_0$ is orthogonal and $R_0$ block upper
triangular, with each block corresponding to an eigenvalue (pair) of $A_0^{-1}$.
Inserting $(Q_0\otimes I)(Q_0^T\otimes I)$ on the left and right of the
matrix yields
%
\begin{align}\label{eq:k3}
\left(R_0^{-1} \otimes \begin{bmatrix} M \\ & \mathbf{0}\end{bmatrix}
	- \delta t I_s \otimes \begin{bmatrix} L_{uu} & L_{up} \\ L_{pu} & S \end{bmatrix} \right)
	(R_0^{-1}Q_0^T\otimes I)\mathbf{K} & = (Q_0^T\otimes I)\mathbf{F}.
\end{align}
%

Thus the fully implicit Runge-Kutta update requires the solution of the
block upper triangular system in \eqref{eq:k3}, which can be accomplished
through backward substitution. There will be two types of diagonal systems
we must solve. The first corresponds to the (typically single) real eigenvalue
$\eta$ of $A_0^{-1}$, for which we must solve an equivalent system as one
obtained for, e.g., backward Euler with \eqref{eq:problem2}:
%
\begin{align}\label{eq:R1}
\begin{bmatrix} \eta M - \delta tL_{uu} & -\delta tL_{up} \\ -\delta tL_{pu} & -\delta tS
	\end{bmatrix}
	\begin{bmatrix} \hat{\mathbf{k}}_j \\ \hat{\ell}_j\end{bmatrix} =
	\begin{bmatrix} \hat{\mathbf{f}}_j \\ \hat{\mathbf{g}}_j \end{bmatrix}.
\end{align}
%
This system is straightforward assuming reasonable $\eta$ values, and
follows directly as one would apply a preconditioner to standard BE or
SDIRK integration schemes.

Complex eigenvalues $\eta + \mathrm{i}\beta$ result in a $2\times 2$ block
in $R_0$, where each block is also a $2\times 2$ block corresponding to the
time-dependent variable and algebraic constraint,
%
\begin{align}\label{eq:R2x2}
\begin{bmatrix}
	\eta M - \delta tL_{uu} & -\delta tL_{up} & \phi M & 0 \\
		-\delta tL_{pu} & -\delta tS & 0 & 0 \\
	-\frac{\beta^2}{\phi}M & 0 & \eta M - \delta tL_{uu} & -\delta tL_{up} \\
	0 & 0 & -\delta tL_{pu} & -\delta tS
	\end{bmatrix}
	\begin{bmatrix} \hat{\mathbf{k}}_j \\ \hat{\ell}_j \\
		\hat{\mathbf{k}}_{j+1} \\ \hat{\ell}_{j+1} \end{bmatrix} =
	\begin{bmatrix} \hat{\mathbf{f}}_j \\ \hat{\mathbf{g}}_j \\
		\hat{\mathbf{f}}_{j+1} \\ \hat{\mathbf{g}}_{j+1} \end{bmatrix}.
\end{align}
%
There are two ways to think about preconditioning \eqref{eq:R2x2}. In 
its current form, we have $2\times 2$ diagonal blocks given as the the
system for a single time step or real eigenvalue, as in \eqref{eq:R1},
and relatively weak off-diagonal coupling. We could directly apply a
block triangular preconditioner, or try some better approximation to
the Schur complement,
%
\begin{align*}
\begin{bmatrix} \eta M - \delta tL_{uu} & -\delta tL_{up} \\ -\delta tL_{pu} & -\delta tS
		\end{bmatrix} +
	\beta^2 \begin{bmatrix} M & 0 \\ 0 & 0 \end{bmatrix}
	\begin{bmatrix} \eta M - \delta tL_{uu} & -\delta tL_{up} \\ -\delta tL_{pu} & -\delta tS
	\end{bmatrix}^{-1}
	\begin{bmatrix} M & 0 \\ 0 & 0 \end{bmatrix}
\end{align*}
%
If we just precondition with the diagonal blocks, the preconditioned 
Schur complement looks like
%
\begin{align*}
I + \frac{\beta^2}{\eta^2} \left(\begin{bmatrix} \eta M - \delta tL_{uu} & -\delta tL_{up} \\
		-\delta tL_{pu} & -\delta tS \end{bmatrix}^{-1}
	\begin{bmatrix} \eta M & 0 \\ 0 & 0 \end{bmatrix} \right)^2.
\end{align*}
%
If the spatial operator is symmetric, the error term should have real-valued
positive eigenvalues. The magnitude is a bit trickier - for the ODE case,
by virtue of a stable time-propagation scheme, we were able to show that
the error term is $\leq 1$ in a field-of-values sense, largely on the
assumption that the spatial operator is negative in a field-of-values
sense. With the algebraic constraint, the spatial operator is likely
indefinite, which raises the important question as to the field-of-values
of $\begin{bmatrix} \eta M - \delta tL_{uu} & -\delta tL_{up} \\
-\delta tL_{pu} & -\delta tS \end{bmatrix}^{-1}$. If this operator
has modes $\approx 0$, then the preconditioning is not so good. If we
can show this is bounded nicely away from 0, we are okay.

Alternatively, we can reorder \eqref{eq:R2x2} to take the form
%
\begin{align}\label{eq:R2x2_2}
\begin{bmatrix}
	\eta M - \delta tL_{uu} & \phi M & -\delta tL_{up} & 0 \\
	-\frac{\beta^2}{\phi}M & \eta M - \delta tL_{uu} & 0 & -\delta tL_{up} \\
	-\delta tL_{pu} & 0 & -\delta tS & 0 \\
	0 & -\delta tL_{pu} & 0 & -\delta tS \\
	\end{bmatrix}
	\begin{bmatrix} \hat{\mathbf{k}}_j \\ \hat{\mathbf{k}}_{j+1} \\ 
		\hat{\ell}_j \\\hat{\ell}_{j+1} \end{bmatrix} =
	\begin{bmatrix} \hat{\mathbf{f}}_j \\ \hat{\mathbf{f}}_{j+1} \\
		 \hat{\mathbf{g}}_j \\ \hat{\mathbf{g}}_{j+1} \end{bmatrix}.
\end{align}
%
The leading $2\times 2$ block is exactly what you get applying a similar
algorithm to a time-dependent diffusion or advection-diffusion problem.
We proved that a block-triangular preconditioner guarantees fast convergence
with Krylov. Thus we have a preconditioner that we know will be effective
for the (1,1)-block. Then we need to precondition the (2,2) Schur complement,
%
\begin{align}\label{eq:blockS}
\begin{bmatrix} -\delta tS & 0 \\ 0 & -\delta tS\end{bmatrix} 
	- \begin{bmatrix} \delta tL_{pu} & 0 \\ 0 & \delta tL_{pu}\end{bmatrix}
	\begin{bmatrix} \eta M - \delta tL_{uu} & \phi M \\
		-\frac{\beta^2}{\phi}M & \eta M - \delta tL_{uu} \end{bmatrix}^{-1}
	\begin{bmatrix} \delta tL_{up} & 0 \\ 0 & \delta tL_{up} \end{bmatrix}.
\end{align}
%
\tcb{Check sign on $S$-term, maybe should be positive.} Here we consider
the approximation
%
\begin{align}\label{eq:blockS_approx}
\begin{bmatrix} -\delta tS & 0 \\ 0 & -\delta tS\end{bmatrix} 
	- \begin{bmatrix} \delta tL_{pu} & 0 \\ 0 & \delta tL_{pu}\end{bmatrix}
	\begin{bmatrix} \eta M - \delta tL_{uu} & 0 \\
		0 & \eta M - \delta tL_{uu} \end{bmatrix}^{-1}
	\begin{bmatrix} \delta tL_{up} & 0 \\ 0 & \delta tL_{up} \end{bmatrix}.
\end{align}
%
This becomes block-diagonal, where each diagonal block can naturally be
preconditioned using standard techniques for preconditioning a single
time step. Theory suggests the approximation in \eqref{eq:blockS_approx}
is good because
%
\begin{align}\label{eq:diag_prec}
\begin{bmatrix} \eta M - \delta tL_{uu} & 0 \\
		0 & \eta M - \delta tL_{uu} \end{bmatrix}^{-1}&
		\begin{bmatrix} \eta M - \delta tL_{uu} & \phi M \\
		-\frac{\beta^2}{\phi}M & \eta M - \delta tL_{uu} \end{bmatrix} \\
& = I + \begin{bmatrix} 0 & \phi(\eta M - \delta tL_{uu})^{-1} M \nonumber\\
		-\frac{\beta^2}{\phi}(\eta M - \delta tL_{uu})^{-1}M & 0 \end{bmatrix} \\
& = I + \begin{bmatrix} 0 & \frac{\phi}{\eta} \\ -\frac{\beta^2}{\phi\eta} & 0 \end{bmatrix}
	\otimes \left( M - \tfrac{\delta t}{\eta} L_{uu}\right)^{-1}M \nonumber\\
& \sim I + \begin{bmatrix} 0 & \frac{\phi}{\eta} \\ -\frac{\beta^2}{\phi\eta} & 0 \end{bmatrix}
	\otimes M^{1/2}\left( M - \tfrac{\delta t}{\eta} L_{uu}\right)^{-1}M^{1/2} \nonumber\\
& = I + \begin{bmatrix} 0 & \frac{\phi}{\eta} \\ -\frac{\beta^2}{\phi\eta} & 0 \end{bmatrix}
	\otimes \left( I - \tfrac{\delta t}{\eta} M^{-1/2}L_{uu}M^{-1/2}\right)^{-1}.\nonumber
\end{align}
%
Suppose that $M^{-1/2}L_{uu}M^{-1/2}$ is symmetric negative definite as in, e.g., Stokes.
Note that the eigenvalues of the $2\times 2$ scalar matrix above are $\pm\beta^2/\eta^2$.
Then, by nature of the Kronecker product, the eigenvalues of \eqref{eq:diag_prec} have
a real part of one and lie on a line in the complex plane between $1\pm\frac{\beta^2}{\eta^2}$.

For problems with nontrivial advection (Navier Stokes/Euler), the eigenvalues
of $I - \tfrac{\delta t}{\eta} M^{-1/2}L_{uu}M^{-1/2}$ likely have a complex component.
This is a little trickier to analyze. Let $x + \mathrm{i}y$ be a complex eigenvalue
of $-M^{-1/2}L_{uu}M^{-1/2}$. Note that $\delta tM^{-1} \approx \delta t/h^2 := C$.
Eigenvalues are then given as
%
\begin{align*}
1 \pm \mathrm{i}\frac{\beta^2}{\eta^2} \cdot
	\frac{1}{1 + \frac{C}{\eta}x \pm \mathrm{i}\frac{C}{\eta}y}
& = 1 \pm \mathrm{i}\frac{\beta^2}{\eta^2} \cdot
	\frac{1 + \frac{C}{\eta}x \mp \mathrm{i}\frac{C}{\eta}y}{(1 + \frac{C}{\eta}x)^2 +
		(\frac{C}{\eta}y)^2} \\
& = 1 \pm \frac{\beta^2}{\eta^2} \cdot
		\frac{\frac{C}{\eta}y}{(1 + \frac{C}{\eta}x)^2 + (\frac{C}{\eta}y)^2}
	\pm \mathrm{i}\frac{\beta^2}{\eta^2} \cdot
	\frac{1 + \frac{C}{\eta}x}{(1 + \frac{C}{\eta}x)^2 +
		(\frac{C}{\eta}y)^2}
\end{align*}
%

% Matlab plot of real part -- want B^2/n^2 times this to be << 1 to stay
% nicely positive and bounded away from zero in a field-of-values sense
% If real part >= imaginary part of eigenvalue, then the preconditioned
% eigenvalues stay nicely bounded positive and away from zero. If there
% are complex eigenvalues with dominant imaginary part c roughly in
% [0.3,3], the diagonal approximation may not be as effective.
%
% f = @(x,y) y ./ ((1+x).^2 + y.^2);
% y0 = 0:0.01:5;                       
% x0 = 0:0.01:5;
% [X,Y] = meshgrid(x0,y0);
% surf(X,Y,f(X,Y), 'EdgeColor', 'none')
% xlabel('Re');
% ylabel('Im');






% ---------------------------------------------------------------------------------------------- %
% ---------------------------------------------------------------------------------------------- %
% ---------------------------------------------------------------------------------------------- %
\newpage
\section{Preconditioning fully implicit IRK system}

Also, I think we may have stumbled upon a way to precondition the fully time
dependent system. If we return to \eqref{eq:keq} for $\mathcal{L}_i\neq\mathcal{L}_j$
and pull out and apply the real Schur decomposition, we have
\begin{align}\label{eq:keq2}
Q_0\otimes I\left( R_0^{-1}\otimes M - \delta t (Q_0^T\otimes I)
	\begin{bmatrix} \mathcal{L}_1  & \\ & \ddots \\ && \mathcal{L}_s\end{bmatrix}
	(Q_0\otimes I)\right)Q_0^T\otimes I.
\end{align}
%
Recall $Q_0$ is orthogonal, so for $\mathcal{L}_i = \mathcal{L}_j$, the coefficients
of $Q_0$ work out so that the off-diagonals of the triple product with $\delta t$
are zero and the diagonals are $\mathcal{L}$. Here instead we have elements with a
form $\sum_{k=1}^s \zeta_k\mathcal{L}_k$, where $\sum_{k=1}^s \zeta_k$ equals one
or zero for diagonal and off-diagonal, respectively. We could either ignore the error
that induces and use a quasiNewton approach inverting \eqref{eq:keq2}, or use
\eqref{eq:keq2} as a linear preonditioner to solve the true Jacobian each step.




\end{document}

