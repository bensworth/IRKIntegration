\documentclass[12pt]{letter}
\pagestyle{plain}
\sloppy

% Reduce space between \closing{ } and names
\makeatletter
\renewcommand{\closing}[1]{\par\nobreak\vspace{\parskip}%
  \stopbreaks
  \noindent
  \ifx\@empty\fromaddress\else
  \hspace*{\longindentation}\fi
  \parbox{\indentedwidth}{\raggedright
       \ignorespaces #1\\[2\medskipamount]%
       \ifx\@empty\fromsig
           \fromname
       \else \fromsig \fi\strut}%
   \par}
\makeatother
% 

\usepackage[headheight=200pt,headsep=200pt,left=1in,right=1in,top=0.8in,bottom=0.8in]{geometry}
\renewcommand{\baselinestretch}{1.0}
\addtolength{\headheight}{-10pt}
\addtolength{\textheight}{0pt}

\address{
Ben Southworth \\
Theoretical Division\\
Los Alamos National Laboratory\\
Los Alamos, NM 87544
\vspace{-5ex}
}
\date{}


\signature{Ben Southworth, Oliver Krzysik, Will Pazner, and Hans De Sterck}

\begin{document}
\vspace{-5ex}
\begin{letter}
\\
\opening{Dear Editor:}
\vspace{3ex}

We are writing to submit the manuscript
\begin{quote}
\emph{Fast parallel solution of fully implicit Runge-Kutta and discontinuous
  Galerkin in time for numerical PDEs, Part I: the linear setting},\\
by Ben Southworth, Oliver Krzysik, Will Pazner, and Hans De Sterck
\end{quote}
for publication in the SIAM Journal on Scientific Computing (SISC).

This is the first paper of a pair that we are submitting to SISC simultaneously
focused on fast parallel solution of fully implicit Runge-Kutta (IRK) methods applied
to numerical PDEs. IRK methods have many benefits over
their more commonly used counterparts such as diagonally implicit Runge-Kutta (DIRK)
methods or BDF methods, but are rarely used in practice due to the difficulty
in solving the stage equations. This paper develops fast, efficient methods to
solve for the Runge-Kutta update when IRK methods are applied to linear PDEs.
Part II uses similar motivation to develop a different and more general
algorithm for nonlinear PDEs and DAEs. The linear setting deserves separate treatment
from nonlinear PDEs (Part II) because by posing the stage equations as an
update in time, the update can be solved for directly, rather than solving for each
stage individually. Compared with the necessary structure for nonlinear
PDEs, the linear algorithm requires significantly less memory, achieves
better conditioning of the ``inner'' preconditioned systems, and is amenable
to three-term Krylov recursion when the underlying spatial discretization is.

Although IRK methods have seen significant research since the 1980s, few
papers have considered the efficient use of IRK methods for high-performance
simulation of numerical PDEs. We believe SISC is the appropriate journal for
this publication because it combines the development of both novel theory and
algorithms, making the use of IRK methods tractable, largely using existing
software infrastructure, and demonstrates the methods on multiple different
of linear PDEs and spatial discretizations.

The article is original and has not been published in other forms. Please
contact us if further information is needed. We look forward to the
review.\\

\closing{Best Regards,}


\end{letter}
\end{document}
