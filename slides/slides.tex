\pdfminorversion=7
\documentclass[blue]{beamer}
\usetheme{Hannover}
\usecolortheme{whale}
\useoutertheme{infolines}
\setbeamertemplate{headline}{} % removes the headline that infolines inserts


% \usefonttheme[onlylarge]{structurebold}
% \setbeamerfont*{frametitle}{size=\normalsize,series=\bfseries}
% \setbeamertemplate{navigation symbols}{}

\usepackage{xcolor}
% This doesn't work yet, trying to figure out how to get beamer color theme and set in title
% \setbeamercolor{frametitle}{bg=\usebeamercolor[bg]{palette primary}}




%\mode<presentation>{
% \usetheme{Darmstadt}

% font themes: default, professionalfonts, serif, structurebold, structureitalicserif, structuresmallcapsserif
\usepackage[T1]{fontenc}
\usepackage[english]{babel}
\usepackage{amsmath,amssymb,amsthm,amsfonts, stmaryrd}
\usepackage{graphicx}
\usepackage[font={small,it}]{caption}
\usepackage{scrextend}
\usepackage{amsfonts}
\usepackage{array}
\usepackage{enumerate}
\usepackage{arydshln}	% Dotted lines in arrays
\usepackage{blkarray}	% Label outside matrix columns
\usepackage{multirow}
\usepackage{adjustbox}

\usepackage[absolute,overlay]{textpos}

\setbeamertemplate{theorems}[numbered]
\newcommand{\todo}[1]{\textcolor{red}{[TODO\@: #1]}}
\newcommand{\tcr}[1]{\textcolor{red}{#1}}
\newcommand{\tcb}[1]{\textcolor{blue}{#1}}


\usepackage{relsize}
\newcommand{\bigqm}[1][1]{\text{\larger[#1]{\textbf{?}}}}

\newcommand*\circled[1]{\tikz[baseline=(char.base)]{
            \node[shape=circle,draw,inner sep=1pt] (char) {#1};}}


\begin{document}

\title[]{Fully implicit Runge-Kutta}
\author[]{Ben Southworth, Oliver Krzysik, Will Pazner, Hans de Sterk
}
\date[]{\vspace{.15in}June 9, 2020}
\frame{\titlepage}


%%===========================================================================


% Show full table of contents after title page
% \begin{frame}<beamer>
% 	\frametitle{Outline}
% 	\tableofcontents
% \end{frame}

%--------------------------------------------------------------------------------%
%--------------------------------------------------------------------------------%
%--------------------------------------------------------------------------------%



% \section{Convergence theory} 

%--------------------------------------------------------------------------------%

\begin{frame}
\frametitle{Why Runge-Kutta}

General Runge-Kutta:
\begin{itemize}
    \item Multi-stage $\mapsto$ easy implementation.
    \item Good stability properties.
\end{itemize}
\vspace{2ex}
\uncover<2->{
Fully implicit Runge-Kutta:
\begin{itemize}
    \item High-order (up to $2s$ for $s$ stages).
    \item High stage-order (for nonlinear PDEs/DAEs).
    \item Can compute pairs of stages in parallel.
    \item Equivalent to DG in time.
\end{itemize}
}
\vspace{2ex}
\uncover<2->{
Assumptions:
\begin{itemize}
    \item For Butcher tableaux $A_0$, $A_0+A_0^T$ is SPD.
    \item Linearized operator $\mathcal{L}$ satisfies
    $\langle\mathcal{L}\mathbf{x},\mathbf{x}\rangle \leq 0$.
\end{itemize}

}

\end{frame}


%--------------------------------------------------------------------------------%

\begin{frame}
\frametitle{Runge-Kutta}

Method of lines for PDEs:
\begin{align*}
    M\mathbf{u}'(t) & =  \mathcal{N}(\mathbf{u},t) \hspace{7.25ex}\text{in }(0,T], \quad \mathbf{u}(0) = \mathbf{u}_0.
\end{align*}
%
Runge-Kutta:
%
\begin{align*}
\mathbf{u}_{n+1} & = \mathbf{u}_n + \delta t \sum_{i=1}^s b_i\mathbf{k}_i, 
    \hspace{5ex}\textnormal{where}\\
M\mathbf{k}_i & = \mathcal{N}\bigg(\mathbf{u}_n + \delta t\sum_{j=1}^s a_{ij}\mathbf{k}_j,
    t_n+\delta tc_i\bigg).\label{eq:stages}
\end{align*}
%

\uncover<2->{
Linear(ized) system:
%
\only<1-2>{
\begin{align*}
\left( \begin{bmatrix} M  & & \mathbf{0} \\ & \ddots \\ \mathbf{0} & & M\end{bmatrix}
    - \delta t \begin{bmatrix} a_{11}\mathcal{L}_1 & ... & a_{1s}\mathcal{L}_1 \\
    \vdots & \ddots & \vdots \\ a_{s1}\mathcal{L}_s & ... & a_{ss} \mathcal{L}_s \end{bmatrix} \right)
    \begin{bmatrix} \mathbf{k}_1 \\ \vdots \\ \mathbf{k}_s \end{bmatrix} 
& = \begin{bmatrix} \mathbf{f}_1 \\ \vdots \\ \mathbf{f}_s \end{bmatrix}.
\end{align*}
}
\only<3->{
\begin{align*}
\left( \begin{bmatrix} I  & & \mathbf{0} \\ & \ddots \\ \mathbf{0} & & I\end{bmatrix}
    - \begin{bmatrix} a_{11}\widehat{\mathcal{L}}_1 & ... & a_{1s}\widehat{\mathcal{L}}_1 \\
    \vdots & \ddots & \vdots \\ a_{s1}\widehat{\mathcal{L}}_s & ... & a_{ss} \widehat{\mathcal{L}}_s \end{bmatrix} \right)
    \begin{bmatrix} \mathbf{k}_1 \\ \vdots \\ \mathbf{k}_s \end{bmatrix} 
& = \begin{bmatrix} \hat{\mathbf{f}}_1 \\ \vdots \\ \hat{\mathbf{f}}_s \end{bmatrix}.
\end{align*}
}
}

\end{frame}

%--------------------------------------------------------------------------------%

\begin{frame}
\frametitle{Linearization}

Step 1:
%
\begin{align*}
\left( A_0^{-1}\otimes I - \begin{bmatrix} \widehat{\mathcal{L}}_1  & \\ & \ddots \\ && \widehat{\mathcal{L}}_s\end{bmatrix}\right)
    (A_0\otimes I)  \begin{bmatrix} \mathbf{k}_1 \\ \vdots \\ \mathbf{k}_s \end{bmatrix}
& = \begin{bmatrix} \mathbf{f}_1 \\ \vdots \\ \mathbf{f}_s \end{bmatrix}.
\end{align*}
%
\uncover<2->{
\noindent
Step 2 (for real Schur decomposition $A_0^{-1} = Q_0R_0Q_0^T$):

{\footnotesize
\begin{align*}
\left( R_0\otimes I - (Q_0^T\otimes I) \begin{bmatrix}
    \widehat{\mathcal{L}}_1  & \\ & \ddots \\ && \widehat{\mathcal{L}}_s\end{bmatrix}
    (Q_0\otimes I)\right) (R_0^{-1}Q_0^T\otimes I) \mathbf{k}\
= (Q_0^T\otimes I)\mathbf{f}.
\end{align*}
}
}
\end{frame}

%--------------------------------------------------------------------------------%

\begin{frame}
\frametitle{Linear solvers}

Must solve $(\eta > 0)$
%
\begin{align*}
\begin{bmatrix} \eta I - \widehat{\mathcal{L}}_1 & \phi I\\
-\frac{\beta^2}{\phi} I & \eta I - \widehat{\mathcal{L}}_2\end{bmatrix},
\end{align*}
%
Convergence of Krylov defined by preconditioned Schur complement,
%
\begin{align*}
S :& = \eta I - \widehat{\mathcal{L}}_2 + \beta^2 (\eta I - \widehat{\mathcal{L}}_1)^{-1} \\
& = \left( (\eta I - \widehat{\mathcal{L}}_2) (\eta I - \widehat{\mathcal{L}}_1)
    + \beta^2 I \right)(\eta I - \widehat{\mathcal{L}}_1)^{-1}.
\end{align*}
%
\uncover<2-2>{
Consider general preconditioned operator
%
\begin{align*}
\mathcal{P}_{\delta,\gamma} & :=
    \Big[(\eta I - \widehat{\mathcal{L}}_2)(\eta I - \widehat{\mathcal{L}}_1) + \beta^2 I\Big]
        (\delta I - \widehat{\mathcal{L}}_1)^{-1}(\gamma I - \widehat{\mathcal{L}}_2)^{-1}.
\end{align*}
}
\end{frame}

%
\begin{frame}
\frametitle{Theory}

%
\begin{theorem}[Optimal preconditioning]
Suppose $\eta > 0$ and $W(\widehat{\mathcal{L}}) \leq 0$,
and suppose $\widehat{\mathcal{L}}$ is real-valued.
Let $\kappa({\cal P}_{\delta, \gamma})$ denote the two-norm condition number of
${\cal P}_{\delta,\gamma}$, for $\delta, \gamma \in (0, \infty)$,
and define $\gamma_*$ by
\begin{align*}
\gamma_* := \frac{\eta^2+\beta^2}{\delta}.
\end{align*}
Then
\begin{align*}
\kappa(\mathcal{P}_{\delta, \gamma_*}) \leq \frac{1}{2 \eta} \left( \delta + \frac{\eta^2 + \beta^2}{\delta} \right).
\end{align*}

Moreover, (i) the bound is tight in the sense that $\exists$ $\widehat{\mathcal{L}}$
that satisfies the bound exactly, and (ii) $\gamma = \gamma_*$ is optimal
in the sense that, without further assumptions on $\widehat{\mathcal{L}}$, $\gamma_*$ minimizes a tight
upper bound on $\kappa({\cal P}_{\delta, \gamma})$ over all $\gamma \in (0, \infty)$.
\end{theorem}
%

\end{frame}

%
\begin{frame}
\frametitle{Linear case, $\mathcal{L}_1=\mathcal{L}_2$, $\gamma ,\delta = $ ?}

\begin{corollary}[Optimal preconditioning with $\gamma = \delta$]
A tight upper bound on the condition number of ${\cal P}_{\delta,\gamma}$
is minimized over $\delta, \gamma \in (0, \infty)$ with $\delta = \gamma = 
\gamma_* = \sqrt{\eta^2 + \beta^2}.$
Furthermore, the condition number of ${\cal P}_{\gamma_*}$
is tightly bounded via
\begin{align*}
\kappa({\cal P}_{\gamma_*}) \leq \sqrt{1 + \frac{\beta^2}{\eta^2}}.
\end{align*}
\end{corollary}

{
\renewcommand{\tabcolsep}{3pt}
\renewcommand{\arraystretch}{1.15}
\begin{table}[!ht]
  \centering
  \begin{tabular}{| c | c | cc | cc | ccc |}  % chktex 44
  \hline
\multirow{2}{*}{Stages} & 2 & \multicolumn{2}{c}{3} & \multicolumn{2}{|c}{4} & \multicolumn{3}{|c|}{5} \\

& {$\lambda_{1,2}^\pm$} & {$\lambda_1$} & {$\lambda_{2,3}^\pm$} & {$\lambda_{1,2}^\pm$} &
    {$\lambda_{3,4}^\pm$} & {$\lambda_1$} & {$\lambda_{2,3}^\pm$} & {$\lambda_{4,5}^\pm$} \\
\hline
Gauss & 1.15 & 1.00 & 1.38 & 1.61 & 1.04 & 1.00 & 1.83 & 1.13 \\
Radau & 1.22 & 1.00 & 1.51 & 1.79 & 1.05 & 1.00 & 2.05 & 1.15 \\
Lobatto & 1.41 & 1.00 & 1.79 & 2.12 & 1.06 & 1.00 & 2.42 & 1.17 \\\hline
  \end{tabular}
  \caption{Bounds on $\kappa(\mathcal{P}_{\gamma_*})$ for Gauss, Radau IIA,
  and Lobatto IIIC.}
\end{table}
}

\end{frame}

%
\begin{frame}
\frametitle{Nonlinear case, $\mathcal{L}_1=\mathcal{L}_2$, $\delta = \eta$}

\begin{corollary}[Optimal preconditioning with $\gamma = \delta$]
A tight upper bound on the condition number of ${\cal P}_{\eta,\gamma}$
is minimized over $\gamma \in (0, \infty)$ with $\delta = \gamma = 
\gamma_* =  \eta + \frac{\beta^2}{\eta}$.
Furthermore, the condition number of ${\cal P}_{\gamma_*}$
is tightly bounded via
\begin{align*}
\kappa({\cal P}_{\gamma_*}) \leq 1 +  \frac{1}{2} \frac{\beta^2}{\eta^2}.
\end{align*}
\end{corollary}

{
\renewcommand{\tabcolsep}{3pt}
\renewcommand{\arraystretch}{1.15}
\begin{table}[!ht]
  \centering
  \begin{tabular}{| c | c | cc | cc | ccc |}  % chktex 44
  \hline
\multirow{2}{*}{Stages} & 2 & \multicolumn{2}{c}{3} & \multicolumn{2}{|c}{4} & \multicolumn{3}{|c|}{5} \\

& {$\lambda_{1,2}^\pm$} & {$\lambda_1$} & {$\lambda_{2,3}^\pm$} & {$\lambda_{1,2}^\pm$} &
    {$\lambda_{3,4}^\pm$} & {$\lambda_1$} & {$\lambda_{2,3}^\pm$} & {$\lambda_{4,5}^\pm$} \\
\hline
Gauss  & 1.17 & 1.00 & 1.46 & 1.80 & 1.05 & 1.00 & 2.18 & 1.14 \\
Radau IIA  & 1.25 & 1.00 & 1.65 & 2.11 & 1.06 & 1.00 & 2.60 & 1.16 \\
Lobatto IIIC & 1.50 & 1.00 & 2.11 & 2.76 & 1.07 & 1.00 & 3.44 & 1.19 \\\hline
  \end{tabular}
  \caption{Bounds on $\kappa(\mathcal{P}_{\gamma_*})$ for Gauss, Radau IIA,
  and Lobatto IIIC.}
\end{table}
}

\end{frame}




\end{document}