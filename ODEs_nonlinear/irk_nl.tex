% \documentclass[a4paper,10pt]{article}
\documentclass[review]{siamart}
\usepackage{url}
\usepackage{amssymb}
\usepackage{amsmath}
\usepackage{bm}
\usepackage{stmaryrd}
\usepackage{array}
\usepackage{multirow}
\usepackage{empheq}
\usepackage{enumitem}
	\setlist{nosep} % or \setlist{noitemsep} to leave space around whole list
\usepackage{color}
\usepackage{verbatim}
\usepackage{showlabels}
\usepackage{adjustbox}
\usepackage{hyperref}
\definecolor{darkgreen}{rgb}{0.0, 0.5, 0}
\usepackage[numbers,sort]{natbib}
\usepackage{cleveref}
\usepackage[belowskip=-5pt]{subcaption}

\newsiamremark{remark}{Remark}
\newsiamremark{assumption}{Assumption}


% \newtheorem{lemma}{Lemma}
% \newtheorem{definition}{Definition}
% \newtheorem{theorem}{Theorem}
% \newtheorem{corollary}{Corollary}

\newcommand{\tcb}{\textcolor{blue}}
\newcommand{\tcp}{\textcolor{purple}}
\newcommand{\todo}[1]{\textcolor{red}{[TODO\@: #1]}}

\newcommand{\mdet}{\operatorname{det}}
\newcommand{\madj}{\operatorname{adj}}

% ------------------------------------------------------------------------------------ %
% ------------------------------------------------------------------------------------ %

\newcommand{\TheTitle}{Fast parallel solution of fully implict Runge-Kutta and discontinous
	Galerkin in time for numerical PDEs, Part II: the nonlinear setting}
\newcommand{\TheAuthors}{B.S. Southworth, O. Krzysik, and W. Pazner}
\headers{Parallel solution of fully implict Runge-Kutta and DG in time I}{\TheAuthors}
\title{{\TheTitle}\thanks{This research was conducted ...
  }}

\author{Ben S. Southworth\thanks{Department of Applied Mathematics,
    University of Colorado,
    U.S.A. (\url{ben.s.southworth@gmail.com}),
    \url{http://orcid.org/0000-0002-0283-4928}}
    \and
    Oliver Krzysik
  	\thanks{School of Mathematical Sciences, Monash University,
  	Australia (\url{oliver.krzysik@monash.edu})}
  	\and
  	Will Pazner\thanks{Lawrence Livermore National Laboratory,
    U.S.A. (\url{pazner1@llnl.gov})}
}

\ifpdf%
\hypersetup{%
  pdftitle={\TheTitle},
  pdfauthor={\TheAuthors}
}
\fi

% ------------------------------------------------------------------------------------ %
% ------------------------------------------------------------------------------------ %

\begin{document}
\maketitle
\allowdisplaybreaks

\begin{abstract}

\end{abstract}


% ------------------------------------------------------------------------------------- %
% ------------------------------------------------------------------------------------- %
% ------------------------------------------------------------------------------------- %
\section{Introduction}\label{sec:intro}

% ------------------------------------------------------------------------------------- %
% ------------------------------------------------------------------------------------- %
\subsection{Fully implicit Runge-Kutta}\label{sec:intro:irk}

Consider the method-of-lines approach to the numerical solution of partial differential
equations (PDEs), where we discretize in space and arrive at a system of ordinary
differential equations (ODEs) in time,
%
\begin{align}\label{eq:problem}
	M\mathbf{u}'(t) =  \mathcal{N}(\mathbf{u},t) \quad\text{in }(0,T], \quad \mathbf{u}(0) = \mathbf{u}_0,
\end{align}
%
where $M$ is a mass matrix and $\mathcal{N}\in\mathbb{R}^{N\times N}$ is a discrete, time-dependent, nonlinear operator depending on $t$ and $\mathbf{u}$ (including potential
forcing terms).\footnote{Note, PDEs with an algebraic constraint, for example, the divergence-free
constraint in Navier Stokes, instead yield a differential algebraic equation (DAE), which
requires separate careful treatment and will be the subject of a forthcoming paper.}
Then, consider time propagation using an $s$-stage Runge-Kutta scheme,
characterized by the Butcher tableaux 
%
\begin{align*}
	\renewcommand\arraystretch{1.2}
	\begin{array}
	{c|c}
	\mathbf{c}_0 & A_0\\
	\hline
	& \mathbf{b}_0^T
	\end{array},
\end{align*}
%
with Runge-Kutta matrix $A_0 = (a_{ij})$, weight vector $\mathbf{b}_0^T = (b_1, \ldots, b_s)^T$,
and quadrature nodes $\mathbf{c}_0 = (c_0, \ldots, c_s)$.

Runge-Kutta methods update the solution using a sum over stage vectors,
%
\begin{align}\label{eq:update}
\mathbf{u}_{n+1} & = \mathbf{u}_n + \delta t \sum_{i=1}^s b_i\mathbf{k}_i, \\
M\mathbf{k}_i & = \mathcal{N}\left(\mathbf{u}_n + \delta t\sum_{j=1}^s a_{ij}\mathbf{k}_j, t_n+\delta tc_i\right).\label{eq:stages}
\end{align}
%
For nonlinear PDEs, $\mathcal{N}$ is linearized using, for example, a Newton or a Picard
linearization of the underlying PDE. Let us denote this linearization
${\mathcal{L}}\in\mathbb{R}^{N\times N}$
(or, in the case of a linear PDE, let $\mathcal{L} := \mathcal{N}$).
Expanding, solving for the stages $\mathbf{k}$ as each step in a nonlinear iteration, or
as the update to $\mathbf{u}$ for a linear PDE, can then be expressed as the
solution of the block linear system,
%
\begin{align}\label{eq:k0}
\left( \begin{bmatrix} M  & & \mathbf{0} \\ & \ddots \\ \mathbf{0} & & M\end{bmatrix}
	- \delta t \begin{bmatrix} a_{11}\mathcal{L}_1 & ... & a_{1s}\mathcal{L}_1 \\
	\vdots & \ddots & \vdots \\ a_{s1}\mathcal{L}_s & ... & a_{ss} \mathcal{L}_s \end{bmatrix} \right)
	\begin{bmatrix} \mathbf{k}_1 \\ \vdots \\ \mathbf{k}_s \end{bmatrix} 
& = \begin{bmatrix} \mathbf{f}_1 \\ \vdots \\ \mathbf{f}_s \end{bmatrix}.
\end{align}
%

The difficulty in fully implicit Runge-Kutta methods (which we will denote IRK) lies in
solving the $Ns\times Ns$ block linear system in \eqref{eq:k0}. This paper focuses on the
parallel simulation of numerical PDEs, where $N$ is typically very large
and $\mathcal{L}$ is highly ill-conditioned. In such cases, direct
solution techniques to solve \eqref{eq:k0} are not a viable option, and fast, parallel 
iterative methods must be used. However, IRK methods are rarely employed in practice due
to the difficulties of solving \eqref{eq:k0}. Even for relatively simple
parabolic PDEs where $\mathcal{L}$ is symmetric positive definite, \eqref{eq:k0}
instead yields a large nonsymmetric system with significant block coupling. For
nonsymmetric systems $\mathcal{L}$ that already have variable coupling, fast iterative
methods are even less likely to yield acceptable performance in solving \eqref{eq:k0}.

% ------------------------------------------------------------------------------------- %
% ------------------------------------------------------------------------------------- %
\subsection{Outline}\label{sec:intro:outline}

Nonlinear problems are then addressed in \Cref{sec:nonlinear}. First, a
simplified Newton method is considered in \Cref{sec:nonlinear:simp},
wherein we develop a method to precondition
\eqref{eq:keq} under the assumption that $\mathcal{L}_i=\mathcal{L}_j$ for all
$i,j$. The proposed method is based on the real Schur decomposition, a technique
not new to Runge-Kutta literature \todo{cite}. However, building on theory from
Section \ref{sec:solve}, we are able to \textit{prove} rapid Krylov convergence
on \eqref{eq:keq}
under basic assumptions of stability and having an effective preconditioner for
systems along the lines of $\eta M - \delta t\mathcal{L}$, exactly as would be
used for, e.g., SDIRK methods. Extensions are then introduced for applying IRK
methods with a true Newton or Picard iteration, where $\mathcal{L}_i\neq
\mathcal{L}_j$, in \Cref{sec:nonlinear:gen}. Numerical results for nonlinear
problems are provided in \Cref{sec:numerics_nonlin}, \todo{including ...}


% ------------------------------------------------------------------------------------- %
% ------------------------------------------------------------------------------------- %
\subsection{Why fully implicit and previous work}\label{sec:intro:hist}




% Simplified newton scheme uses fixed Jacobian for all stages
% Single step Newton (Cooper (83/90), Gonzales-Pinto (95/97)
% ---------------- ODEs and LU ---------------- %
% Varah (79):
% \begin{itemize}
% 	\item Much better description of Butcher's method. Transforms coordinate system of Jacobian tensor.
% 	Still relies on this tensor structure to do said transformation. 
% 	\item Transforms Jacobian to Hessenberg form to avoid repeated action computing LU of a shifted 
% 	Jacobian.
% \end{itemize}

% Cooper:
% \begin{itemize}
% 	\item (83,90,93): develops single Newton scheme using SOR or various matrix splittings,
% 	applied to ODEs. This is a single-step
% 	Newton method, where the actual Jacobian solve is replaced with an SOR iteration. Here, Butcher
% 	matrix A is replaced with an approximate matrix with one positive real eigenvalue. 
% \end{itemize}

% Pinto:
% \begin{itemize}
% 	\item (95) Mostly ODEs, do consider 1d-space-1d-time burgers on a very small grid. 
% 	\item (96) additional iteration of single Newton, makes it quasi-Newton like. 
% 	\item (01) Analysis of single Newton (SOR) w/ simplified Newton for higer order. 
% \end{itemize}

% Brugano (2014) and Antonana (2018)
% \begin{itemize}
% 	\item New splitting and IRK techniques for Hamiltonian problems where conservation is
% 	important. Used for ODEs. 
% \end{itemize}


% % ---------------- PDEs ---------------- %

% Jay
% \begin{itemize}
% 	\item 1999: develops a preconditioning technique based on W-transformation. W-transformation
% 	yields a real-valued block-tridiagonal matrix. This can be solved using LU, but the inverses
% 	have a recursive nature (like Schur complements). They precondition w/ an approximate LU,
% 	where each formal inverse in LU is approximated, $H_i = I +
% 	\zeta_{i-1}^2\delta t^2 JH_{i-1}^{-1}J$, for Jacobian $J$ and matrix $H$ that must be
% 	inverted in LU, with $\hat{H}_i = I - \gamma_i\delta tJ$, for a certain \gamma_i$. 

% 	\item 2000: Analyzes simplified Newton to approximate time-dependent Jacobian with
% 	Kronecker product form. 
% \end{itemize}

% Houwen 
% \begin{itemize}
% \item TRIANGULARLY(1997): Uses crout factorization to pick lower triangular preconditioner.
% \item Paralell(97): 
% \end{itemize}

% Hoffman 
% \begin{itemize}
% \item (97): Also uses triangular preocnditioner on the nonlinear level.
% \end{itemize}


% Van Lent (2004)
% \begin{itemize}
% 	\item Multigrid for IRK?
% \end{itemize}

% Staff \& Mardal (2006)
% \begin{itemize}
% 	\item One of first paper to consider preconditioning the fully implicit RK system.
% 	Use block Jacobi and block lower triangular preconditioners for the diffusion equation.
% 	Use multigrid V-cycles and full Newton time-dependent Jacobian. Upper triangular is
% 	bad compared to block Jacobi and lower triangular (this has appeared elsewhere in
% 	literature -- $A_0$ is dominant in lower triangular part -- Crout factorization
% 	in I think Messina or Van der Houwen; comes up again in Brugano (2015)).
% \end{itemize}
% Mardal (2007)
% \begin{itemize}
% 	\item Analyze block-diagonal preconditioners in a Sobolev setting, demonstrate
% 	conditioning of the preconditioned operator to be optimal in the independent of
% 	$h$ sense. Use multigrid w/ diffusion as example. 
% \end{itemize}
% Nilssen (2011)
% \begin{itemize}
% 	\item Analogous to above, Sobolev analysis for block-diagonal preconditioning
% 	applied to the bidomain equations.
% \end{itemize}

% Xie (2011)
% \begin{itemize}
% 	\item Proposes a modified simplified Newton for the time-dependent cast, where the Jacobian is
% 	formed based on a least squares approximation to the true RK coefficients, evaluating all entries
% 	at a single time point. In example problems, modified Jacobian typically converged faster
% 	than simplified (evaluated at previous time step), up to 2x less iterations/time. 
% \end{itemize}

% Hao Chen:
% \begin{itemize}
% 	\item (2014) Develops a splitting iterative method to precondition IRK matrices, similar to ADI schemes.
% 	Proves that for definite spatial operators and Butcher matrices (that is, eignvalues have positive
% 	or negative real parts), $\rho(T) < 1$, where $T$ is the fixed-point iteration matrix. Look at 
% 	diffusion equation with IRK and BVMs.
% 	\item (2016) Analogous to above, extended to wave equation.
% \end{itemize}

% Pazner
% \begin{itemize}
% 	\item
% \end{itemize}

% \begin{itemize}
% 	\item if the underlying IRK methods are A-stable, irreducible, and have invertible coefficientmatrixA, such as Gauss, RadauIIA, and LobattoIIIC (cf., e.g., [Hairer]), then the real partsof the eigenvalues ofAare positive,
% 	\item One benefit of this method is ability to use PCG or GMRES on smaller system, not full system. 
% \end{itemize}

% ------------------------------------------------------------------------------------- %
% ------------------------------------------------------------------------------------- %
\subsection{A preconditioning framework and stability}\label{sec:intro:stab}

%
\begin{align}\label{eq:keq}
\left( A_0^{-1}\otimes M - \delta t \begin{bmatrix} \mathcal{L}_1  & \\ & \ddots \\ && \mathcal{L}_s\end{bmatrix}\right)
	(A_0\otimes I)	\begin{bmatrix} \mathbf{k}_1 \\ \vdots \\ \mathbf{k}_s \end{bmatrix}
& = \begin{bmatrix} \mathbf{f}_1 \\ \vdots \\ \mathbf{f}_s \end{bmatrix}.
\end{align}

%
\begin{assumption}\label{ass:eig}
Assume that all eigenvalues of $A_0$ (and equivalently $A_0^{-1})$ have positive real part.
\end{assumption}


\begin{align}\label{eq:fov}
W(\mathcal{L}) := \left\{ \langle \mathcal{L}\mathbf{x},\mathbf{x}\rangle \text{ : }
	\|\mathbf{x}\| = 1 \right\}.
\end{align}


%
\begin{assumption}\label{ass:fov}
Let $\mathcal{L}$ be the linear spatial operator, and assume that $W(\mathcal{L}) \leq 0$.
\end{assumption}


% ------------------------------------------------------------------------------------- %
% ------------------------------------------------------------------------------------- %
% ------------------------------------------------------------------------------------- %
\section{Linear preconditioning theory}\label{sec:theory}

Let $\zeta_{\pm} := \eta \pm \mathrm{i}\beta$ denote an eigenvalue of $A_0^{-1}$,
where under \Cref{ass:eig} $\eta > 0$. The methods derived in this paper depend on
solving a $2\times 2$ block system along the lines of
%
\begin{align}\label{eq:block0}
\begin{bmatrix} \eta I - \widehat{\mathcal{L}}_1 & \phi I\\
-\frac{\beta^2}{\phi} I & \eta I - \widehat{\mathcal{L}}_2\end{bmatrix},
\end{align}
%
for some $\phi \neq 0$ and where it is assumed that $W(\widehat{\mathcal{L}}_i) \geq 0$
in \Cref{ass:fov} for $i=1,2$. In some
cases (in particular, simplified Newton methods) $\widehat{\mathcal{L}}_1 =
\widehat{\mathcal{L}}_2$, but we are particularly interested in the more general
setting of $\widehat{\mathcal{L}}_1 \neq \widehat{\mathcal{L}}_2$. We will solve
this system using Krylov methods with block lower-triangular preconditioners of the form
%
\begin{equation}\label{eq:Lprec}
L_P := \begin{bmatrix} \eta I - \widehat{\mathcal{L}}_1 & \mathbf{0} \\ -\frac{\beta^2}{\phi} I
	& \widehat{S}\end{bmatrix}^{-1},
\end{equation}
%
where $\widehat{S}$ is some approximation to the Schur complement of \eqref{eq:block0},
which is given by
%
\begin{align}\label{eq:Schur}
S & := \eta I - \widehat{\mathcal{L}}_2 + \beta^2 (\eta I - \widehat{\mathcal{L}}_1)^{-1}.
\end{align}
%

When applying GMRES to block $2\times 2$ operators preconditioned with a lower
(or upper) triangular preconditioner as in \eqref{eq:Lprec}, convergence 
is exactly defined by convergence of GMRES applied to the preconditioned Schur
complement, $\widehat{S}^{-1}S$ \cite{2x2block}. If $\widehat{S} = S$ is exact,
exact convergence on the larger $2\times2$ system is guaranteed in two iterations
(or one iteration with a block LDU). This section focuses on the development of
robust preconditioners for the Schur complement \eqref{eq:Schur}. As a result of
\Cref{ass:fov}, the second term in \eqref{eq:Schur},
$(\eta I - \widehat{\mathcal{L}}_1)^{-1}$ is nicely bounded and conditioned.
To that end, we consider preconditioners of the form
%
\begin{align*}
\widehat{S}_\gamma := \gamma I - \widehat{\mathcal{L}}_2
\end{align*}
%
for some $\gamma \geq \eta$. \Cref{sec:theory:spd} considers the simplest case
of symmetric semi-definite $\widehat{\mathcal{L}}_1 = \widehat{\mathcal{L}}_2$
and uses eigenvalues to determine an optimal choice of $\gamma\mapsto\gamma_*$,
with the resulting preconditioned operator having norm $\mathcal{O}(1)$ with
only weak dependence on number of stages/polynomial order. \Cref{sec:theory:gen}
then extends the theory to prove that cond$(\widehat{S}_{\gamma_*}^{-1}S)
\leq 2 + \tfrac{\beta^2}{\eta^2}$ under minimal assumptions on
$\widehat{\mathcal{L}}_1 \neq \widehat{\mathcal{L}}_2$. 

% ---------------------------------------------------------------------------------------------- %
% ---------------------------------------------------------------------------------------------- %
\subsection{Choosing $\gamma$: the SPD case}\label{sec:theory:spd}

Suppose $-\mathcal{L}$ is SPSD with a spectrum $\subset [0,\infty)$, and
consider preconditioning $S$ with $(\gamma I- \widehat{\mathcal{L}})^{-1}$ for
some $\gamma \neq \eta$. The preconditioned operator then takes the form
%
\begin{align}\nonumber
(\gamma I- \widehat{\mathcal{L}})^{-1}S & = (\gamma I - \widehat{\mathcal{L}})^{-1}
	\left[ (\gamma I - \widehat{\mathcal{L}}) + (\eta-\gamma)I + \beta^2 (\eta I - \widehat{\mathcal{L}})^{-1}\right] \\
& = I - (\gamma - \eta)( \gamma I- \widehat{\mathcal{L}})^{-1} + 
	\beta^2( \gamma I- \widehat{\mathcal{L}})^{-1}
		( \eta I-\widehat{\mathcal{L}})^{-1} \nonumber\\
& = I - \frac{\gamma - \eta}{\gamma} ( I- \tfrac{1}{\gamma}\widehat{\mathcal{L}})^{-1} + 
	\frac{\beta^2}{\gamma\eta}( I- \tfrac{1}{\gamma}\widehat{\mathcal{L}})^{-1}
		( I- \tfrac{1}{\eta}\widehat{\mathcal{L}})^{-1},\label{eq:gamma0}
\end{align}
%
with spectrum given by
%
\begin{align}\label{eq:eig_gamma}
\mathcal{F}(\gamma,\lambda) :&= 
	1 - \frac{\gamma-\eta}{\gamma + \lambda} + \frac{\beta^2}{(\gamma + \lambda)(\eta+\lambda)},
\end{align}
%
where $\lambda\in\sigma(-\mathcal{L})$. If we choose $\gamma > 0$,
\eqref{eq:gamma0} is SPD, and the condition number of \eqref{eq:gamma0} is given by
%
\begin{align}\label{eq:cond0}
\textnormal{cond}\left((\gamma I- \widehat{\mathcal{L}})^{-1}S\right) & =
	\frac{\lambda_{\max}\left((\gamma I- \widehat{\mathcal{L}})^{-1}S\right)}
		{\lambda_{\min}\left((\gamma I- \widehat{\mathcal{L}})^{-1}S\right)}.
\end{align}
%
As $h,\delta t\to 0$, it is typical for parabolic problems that
the spectrum $\lambda\in\sigma(-\widehat{\mathcal{L}})$
becomes increasingly dense in the interval $[0,\infty)$. For such limiting
behavior, the condition number \eqref{eq:cond0} can be expressed
precisely as
%
\begin{align}\label{eq:cond1}
\textnormal{cond}\left((\gamma I- \widehat{\mathcal{L}})^{-1}S\right) &
	\hspace{2ex}\mapsto\hspace{2ex}
	c(\gamma) :=
	\frac{\max_{\lambda\in[0,\infty)} \mathcal{F}(\gamma,\lambda)}
		{\min_{\lambda\in[0,\infty)} \mathcal{F}(\gamma,\lambda)},
\end{align}
%
where $c(\gamma)$ also provides an upper bound on
$\textnormal{cond}\left((\gamma I- \widehat{\mathcal{L}})^{-1}S\right)$ when
$\sigma(-\widehat{\mathcal{L}})\neq [0,\infty)$. The following theorem 
derives a closed form for $c(\gamma)$ and corresponding optimal
(minimizing) value of $\gamma > 0$.

%
\begin{theorem}[Conditioning of preconditioned operator]\label{th:eig}
Let $\lambda\in[0,\infty)$, $0<\eta\leq \gamma$, and $\beta \geq 0$.
Define $\mathcal{F}(\gamma,\lambda)$ as in \eqref{eq:eig_gamma} and
$c(\gamma)$ as in \eqref{eq:cond1}. Then 
\begin{align}\label{eq:gammastar}
\gamma_*  :=  \underset{0 \leq \gamma < \infty}{\textnormal{argmin}} \; c(\gamma)
	= \eta + \frac{\beta^2}{\eta},
\end{align}
where
\begin{align} \label{eq:cond_min}
\textnormal{cond}\left((\gamma_* I- \widehat{\mathcal{L}})^{-1}S\right) \leq
	c(\gamma_*) = \frac{1}{2} \left( 1 + \sqrt{1 + \left( \frac{\beta}{\eta} \right)^2} \right).
\end{align}

\end{theorem}
\begin{proof}
This theorem is based on the minimization problem
%
\begin{align}\label{eq:gam_opt}
\gamma_* & = \textnormal{argmin}_{\gamma \in (0,\infty)}
	\frac{\max_{\lambda\in[0,\infty)} \mathcal{F}(\gamma,\lambda)}
		{\min_{\lambda\in[0,\infty)} \mathcal{F}(\gamma,\lambda)}.
\end{align}
%
The $\gamma = \eta$ case is special, and the resulting simplified form of
\eqref{eq:eig_gamma} quickly yields $c(\eta) = 1+\tfrac{\beta^2}{\eta^2}$.
Thus moving forward we consider $\eta \neq \gamma > 0$.
Minima and maxima in $\lambda$ may be obtained at one of the endpoints,
$\lambda = 0$ or $\lambda\to\infty$, or at a critical point of \eqref{eq:eig_gamma}
in $\lambda$. Taking the partial with respect to $\lambda$, we have
%
\begin{align}\label{eq:partial_l}
\frac{\partial\mathcal{F}}{\partial\lambda} & =
	\frac{(\gamma-\eta)(\eta+\lambda)^2 - \beta^2(\gamma+\eta+2\lambda)}
		{(\gamma+\lambda)^2(\eta+\lambda)^2}.
\end{align}
%
Noting that the denominator is nonnegative for $\eta,\gamma>0$ and $\lambda \geq 0$,
the critical points are obtained at zeros of the numerator in \eqref{eq:partial_l},
which can be written as a quadratic polynomial in $\lambda$:
%
\begin{align*}
(\gamma-\eta)\lambda^2 - 2(\eta^2+\beta^2 - \eta\gamma)\lambda + 
	\gamma(\eta^2-\beta^2) - \eta(\eta^2+\beta^2) = 0.
\end{align*}
%
For $\gamma \neq \eta$ the two real roots are given by
%
\begin{align}\label{eq:roots}
\lambda_{\pm} & := \frac{\beta^2 + \eta^2 - \gamma\eta \pm
	\beta\sqrt{\beta^2 + (\gamma-\eta)^2}}{\gamma-\eta}.
\end{align}
%
Thus, we have four potential points at which a maximum or minimum in $\lambda$
can be achieved, $\{0,\infty, \lambda_\pm\}$, where
%
\begin{align}\label{eq:F_gamma}
\begin{split}
\mathcal{F}(\gamma,\infty) & = 1, \hspace{20ex}
\mathcal{F}(\gamma,\lambda_+) = \frac{2\beta}{\beta + \sqrt{\beta^2 + (\gamma-\eta)^2}}, \\
\mathcal{F}(\gamma,0) & = 1 + \frac{\eta^2+\beta^2-\eta\gamma}{\eta\gamma},
\hspace{4ex}
\mathcal{F}(\gamma,\lambda_-) = \frac{2\beta}{\beta - \sqrt{\beta^2 + (\gamma-\eta)^2}}.
\end{split}
\end{align}
%

Note that $\mathcal{F}(\gamma,\lambda_-) < 0$ for all $\gamma\neq\eta$, which
contradicts the previously stated result that $\mathcal{F}(\gamma,\lambda)$ is SPD
for $\gamma > 0$ and $\lambda\geq 0$, implying $\lambda_- < 0$.
Thus we are left with three critical points to consider, 
$\{0,\lambda_+,\infty\}$. Further, observe that
%
\begin{align*}
\mathcal{F}(\gamma,\lambda_+) = \frac{2\beta}{\beta + \sqrt{\beta^2 + (\gamma-\eta)^2}}
	= \frac{2}{1 + \sqrt{1 + (\gamma-\eta)^2/\beta^2}} < 1 = \mathcal{F}(\gamma,\infty),
\end{align*}
%
for all $\gamma \neq \eta$.
It follows that the maximum of ${\cal F}(\gamma,\lambda)$ w.r.t.
$\lambda$ must thus occur for $\lambda \in \{ 0, \infty \}$. Evaluating
\eqref{eq:F_gamma}, we have
%
\begin{align}\label{eq:eig_max}
\max_{\lambda\in[0,\infty)} \mathcal{F}(\gamma,\lambda) & = 
\begin{cases}
\displaystyle
\mathcal{F}(\gamma,0) 
= 
\frac{1}{\gamma} \left(\eta + \frac{\beta^2}{\eta} \right), 
\quad 
& 0 < \gamma \leq \eta +  \tfrac{\beta^2}{\eta}, \\
\displaystyle
\mathcal{F}(\gamma,\infty) 
= 
1, \quad & \eta + \tfrac{\beta^2}{\eta} \leq \gamma < \infty
\end{cases},
\end{align}
%
with continuity at the special case of $\gamma = \eta$.

For the minimum, because $\mathcal{F}(\gamma,\lambda_+) < \mathcal{F}(\gamma,\infty),
\forall \gamma \neq \eta$, the minimum cannot occur for $\lambda \to \infty$, which
implies the minimum of ${\cal F}(\gamma,\lambda)$ w.r.t. $\lambda$ must occur
at $\lambda \in \{0, \lambda_+\}$. For $\gamma \geq \eta + \beta^2 / \eta$,
${\cal F}(\gamma, \lambda_+) < 1 \leq {\cal F}(\gamma, 0)$. To consider
$0 < \gamma < \eta + \beta^2 / \eta$, define
%
\begin{align*}
{\cal G}(\gamma) = \mathcal{F}(\gamma,0) -  \mathcal{F}(\gamma,\lambda_+),
	\quad 0 < \gamma \leq \eta + \beta^2 / \eta.
\end{align*}
%
To determine which value of ${\lambda} \in \{0, \lambda_+ \}$ minimizes
${\cal F}(\gamma, \lambda)$ we can consider the sign of ${\cal G}$. From
above, we know that ${\cal G} > 0$ for $\gamma = \eta + \beta^2/\eta$, so
we seek roots $\hat{\gamma}\in(0,\eta+\beta^2/\eta]$, such that
${\cal G}(\hat{\gamma}) = 0$, indicating a sign change in ${\cal G}$.

Setting ${\cal G}(\hat{\gamma}) = 0$ and assuming $\beta > 0$, we have
\begin{align}\nonumber
\frac{2\beta}{\beta + \sqrt{\beta^2 + (\hat{\gamma}-\eta)^2}} & =
	\frac{1}{\hat{\gamma}} \left(\eta + \frac{\beta^2}{\eta} \right), \\
\Longleftrightarrow \hspace{5ex} \label{eq:rhs_pf}
2\beta \eta \hat{\gamma} -\beta(\eta^2 + \beta^2)  & = 
	(\eta^2 + \beta^2) \sqrt{\beta^2 + (\hat{\gamma}-\eta)^2}. 
\end{align}
%
Squaring both sides of the latter equation leads to a quadratic equation in $\hat{\gamma}$, 
%
\begin{align*}
z(\hat{\gamma}) = \big[ 4 \beta^2 \eta^2 - (\eta^2 + \beta^2)^2 \big] \hat{\gamma}^2 + 2 \eta (\eta^2 + \beta^2)(\eta^2 - \beta^2) \hat{\gamma} - \eta^2(\eta^2 + \beta^2)^2 = 0.
\end{align*}
%
Note that for $\eta\neq\beta$ the discriminant of $z(\hat{\gamma})$ is zero, implying
$z(\hat{\gamma})$ has one real root, given by
%
\begin{align} \label{eq:gamma_hat}
\hat{\gamma} = \eta \frac{\eta^2 + \beta^2}{\eta^2 - \beta^2}. 
\end{align}
%
If $\eta < \beta$, then $\hat{\gamma} < 0$, while for $\eta > \beta$, we
have $\hat{\gamma} > \eta+\beta^2/\eta$. In both of these cases, $\hat{\gamma}
\not\in(\eta,\eta+\beta^2/\eta]$. For $\eta = \beta$, the discriminant of 
$z$ is $-4\beta^6 < 0$, implying no real roots and, thus, no $\hat{\gamma}$
that satisfies ${\cal G}(\hat{\gamma}) = 0$. Altogether, ${\cal G}$ has no
sign changes in the interval $(\eta,\eta+\beta^2/\eta]$, implying
${\cal F} (\gamma, \lambda_+) < {\cal F}(\gamma, 0)$ for $0 < \gamma
\leq \eta + \beta^2/\eta.$ Combining with the earlier discussion, we have
%
\begin{align*}
\min_{\lambda\in[0,\infty)} \mathcal{F}(\gamma,\lambda) &=
	\mathcal{F}(\gamma,\lambda_+) =
\frac{2\beta}{\beta + \sqrt{\beta^2 + (\gamma-\eta)^2}}
	\quad 0 < \gamma < \infty.
\end{align*}
%
Combining with \eqref{eq:eig_max} and the special case mentioned previously where
$c(\eta) = 1+\tfrac{\beta^2}{\eta^2}$, we have $c(\gamma)$ as a continuous
function of $\gamma>0$,
\begin{align}
\label{eq:condc}
c(\gamma) 
\coloneqq
% \textnormal{cond}\left((\gamma I- \widehat{\mathcal{L}})^{-1}S\right)  
% =
\begin{cases}
\displaystyle
\frac{1}{\gamma} \left(\eta + \tfrac{\beta^2}{\eta} \right) \frac{\beta + \sqrt{\beta^2 + (\gamma-\eta)^2}}{2\beta}
\eqqcolon c_1(\gamma), 
\quad & 0 < \gamma \leq \eta + \tfrac{\beta^2}{\eta},
\\[3ex]
\displaystyle
\frac{\beta + \sqrt{\beta^2 + (\gamma-\eta)^2}}{2\beta}
\eqqcolon c_2(\gamma), 
\quad &\eta + \tfrac{\beta^2}{\eta} \leq \gamma < \infty 
\end{cases}.
\end{align}
%

To minimize $c(\gamma)$, note that $c_2$ is an increasing function over the interval
for which it is defined. Thus, its minimum is achieved at the left boundary
of its domain, with minimum value
%
\begin{align} 
\label{eq:c2_min}
\min_{\gamma\geq\eta} c_2(\gamma) = 
	c_2\left(\eta + \frac{\beta^2}{\eta} \right) =
\frac{1}{2} \left( 1 + \sqrt{1 + \left( \frac{\beta}{\eta} \right)^2} \right).
\end{align}
%
For $c_1$, differentiating and solving for the roots of its derivative
yields the two critical points, $\gamma = \eta$ and $\gamma = \hat{\gamma}$
as in \eqref{eq:gamma_hat}. Note that $\hat{\gamma}\not\in(\eta, \eta + \beta^2/\eta]$
while $c(\eta) = 1+\tfrac{\eta^2}{\beta^2} > c_2(\eta + \beta^2/\eta)$
and $\lim_{\gamma\to 0} c_2(\gamma) = \infty$. It follows that the minimum of $c_2$
will be obtained at $\gamma = \eta + \beta^2/\eta$, and
$\gamma_* := \textnormal{argmin}_{\gamma\geq\eta} c(\gamma) = 
\eta + \tfrac{\beta^2}{\eta}$. Plugging $\gamma = \gamma_*$ into
\eqref{eq:condc} completes the proof.

\end{proof}

% ---------------------------------------------------------------------------------------------- %
% ---------------------------------------------------------------------------------------------- %
\subsection{General conditioning theory}\label{sec:theory:gen}

Now we consider preconditioning \eqref{eq:block0} for general operators 
$\widehat{\mathcal{L}}_1\neq\widehat{\mathcal{L}}_2$ that are not necessarily diagonalizable.
In particular, we consider the $\ell^2$-conditioning of the right-preconditioned
Schur complement
%
\begin{align}\nonumber
\mathcal{P}_\gamma :&=
S(\gamma I- \widehat{\mathcal{L}}_2)^{-1} \\
& = \left[ (\gamma I - \widehat{\mathcal{L}}_2) + (\eta-\gamma)I +
	\beta^2 (\eta I - \widehat{\mathcal{L}}_1)^{-1}\right]
	(\gamma I - \widehat{\mathcal{L}}_2)^{-1} \nonumber\\
% & = I - (\gamma - \eta)( \gamma I- \widehat{\mathcal{L}}_2)^{-1} + 
% 	\beta^2( \eta I-\widehat{\mathcal{L}}_1)^{-1}
% 	( \gamma I- \widehat{\mathcal{L}}_2)^{-1}\nonumber\\
& = I - \frac{\gamma - \eta}{\gamma} ( I- \tfrac{1}{\gamma}\widehat{\mathcal{L}}_2)^{-1} + 
	\frac{\beta^2}{\gamma\eta}( I- \tfrac{1}{\eta}\widehat{\mathcal{L}}_1)^{-1}
	( I- \tfrac{1}{\gamma}\widehat{\mathcal{L}}_2)^{-1},\label{eq:gamma2}
\end{align}
%
\Cref{th:cond} derives an upper bound on condition number of the preconditioned
operator \eqref{eq:gamma2} for $\gamma = \gamma_*$ \eqref{eq:gammastar}.
Note, considering right preconditioning is a theoretical tool to facilitate
the proof of \Cref{th:cond}, but \tcb{numerical results demonstrate effective
left or right preconditioning.} It is also worth discussing the additional
assumption in \Cref{th:cond} that $\langle\widehat{\mathcal{L}}_1\mathbf{w},
\widehat{\mathcal{L}}_2\mathbf{w}\rangle\geq 0$. In general, this is a technical
assumption to simplify the proof, but is not believed to be necessary for the
result to hold. Nevertheless, the assumption holds trivially for
$\widehat{\mathcal{L}}_1=\widehat{\mathcal{L}}_2$, while for 
$\widehat{\mathcal{L}}_1 \neq \widehat{\mathcal{L}}_2$, it is more or less
an assumption that $\widehat{\mathcal{L}}_1$ and $\widehat{\mathcal{L}}_2$
are ``close'' in some sense, a reasonable assumption given they are
defined by the same operator evaluated at a perturbation in time.

%
\begin{theorem}[Conditioning of preconditioned operator]\label{th:cond}
Suppose Assumptions \ref{ass:eig} and \ref{ass:fov} hold, that is, $\eta > 0$
and $W(\widehat{\mathcal{L}}_1),W(\widehat{\mathcal{L}}_2) \leq 0$ \eqref{eq:fov}.
Additionally, assume that $\langle\widehat{\mathcal{L}}_1\mathbf{w},
\widehat{\mathcal{L}}_2\mathbf{w}\rangle\geq 0$. Let $\mathcal{P}_\gamma$
denote the right-preconditioned Schur complement \eqref{eq:gamma2}, with
preconditioner $(\gamma I - \widehat{\mathcal{L}}_2)^{-1}$, for $\gamma >\eta$,
and define $\gamma_* := \tfrac{\eta^2+\beta^2}{\eta}$. Then
\begin{align}\label{eq:gammastar_cond}
\textnormal{cond}(\mathcal{P}_{\gamma_*}) \leq 
	2 + \frac{\beta^2}{\eta^2}.
\end{align}
\end{theorem}
\begin{proof}
Recall for matrix $A$, cond$(A) = \|A\|\|A^{-1}\|$.
First, consider bounding $\|(\gamma I- \widehat{\mathcal{L}}_2)^{-1}S\|$ for
$\gamma \geq \eta$:
%
\begin{align}\nonumber
\|\mathcal{P}_\gamma\| & = \left\| I - \frac{\gamma - \eta}{\gamma}
	( I- \tfrac{1}{\gamma}\widehat{\mathcal{L}}_2)^{-1} + 
	\frac{\beta^2}{\gamma\eta}( I- \tfrac{1}{\eta}\widehat{\mathcal{L}}_1)^{-1}
	( I- \tfrac{1}{\gamma}\widehat{\mathcal{L}}_2)^{-1} \right\| \\
% & \leq \left\| I - 2\frac{\gamma-\eta}
% 	{\gamma}\left(I - \tfrac{1}{\gamma}\widehat{\mathcal{L}}\right)^{-1}\right\| +
% 		\frac{\beta^2 + (\gamma-\eta)^2}{\gamma^2}\left\|
% 		\left(I - \tfrac{1}{\gamma}\widehat{\mathcal{L}}\right)^{-2} \right\| \\
& \leq \left\| I - \frac{\gamma-\eta}
	{\gamma}\left(I - \tfrac{1}{\gamma}\widehat{\mathcal{L}}_2\right)^{-1}\right\| +
		\frac{\beta^2}{\gamma\eta}
		\left\|( I- \tfrac{1}{\eta}\widehat{\mathcal{L}}_1)^{-1} \right\|
		\left\|( I- \tfrac{1}{\gamma}\widehat{\mathcal{L}}_2)^{-1}\right\|\nonumber \\
& \leq \left\| I - \frac{\gamma-\eta}
	{\gamma}\left(I - \tfrac{1}{\gamma}\widehat{\mathcal{L}}_2\right)^{-1}\right\| +
		\frac{\beta^2}{\gamma\eta}. \label{eq:Pgn}
\end{align}
%
For the first term, note that maximizing over $\mathbf{v}\in\mathbb{R}^n$ and
letting $\mathbf{v} := (I - \tfrac{1}{\gamma}\widehat{\mathcal{L}}_2)\mathbf{w}$,
%
\begin{align*}
\left\| I - \tfrac{\gamma-\eta}
	{\gamma}(I - \tfrac{1}{\gamma}\widehat{\mathcal{L}}_2)^{-1}\right\|^2
		& = \sup_{\mathbf{v}\neq\mathbf{0}} \frac{\left\| [I - \frac{\gamma-\eta}
	{\gamma}(I - \tfrac{1}{\gamma}\widehat{\mathcal{L}}_2)^{-1}]\mathbf{v}\right\|^2}{\|\mathbf{v}\|^2}  \\
& = \sup_{\mathbf{w}\neq\mathbf{0}} \frac{\left\| (I - \tfrac{1}{\gamma}\widehat{\mathcal{L}}_2 -
		\frac{\gamma-\eta}{\gamma}I )\mathbf{w}\right\|^2}{\|(I - \tfrac{1}{\gamma}\widehat{\mathcal{L}}_2)
		\mathbf{w}\|^2} \\
% & = \sup_{\mathbf{w}\neq\mathbf{0}} \frac{\left\|[(1 - 2\frac{\gamma-\eta}{\gamma})
% 	I - \tfrac{1}{\gamma}\widehat{\mathcal{L}}_2]\mathbf{w}\right\|^2}{\|(I - \tfrac{1}{\gamma}\widehat{\mathcal{L}}_2)
% 		\mathbf{w}\|^2} \\
& = \sup_{\mathbf{w}\neq\mathbf{0}} \frac{(\tfrac{\eta}{\gamma})^2\|\mathbf{w}\|^2
	- \tfrac{\eta}{\gamma^2}\langle (\widehat{\mathcal{L}}_2 + \widehat{\mathcal{L}}_2^T)
		\mathbf{w},\mathbf{w}\rangle + \tfrac{1}{\gamma}^2\|\widehat{\mathcal{L}}_2\mathbf{w}\|^2}
	{\|\mathbf{w}\|^2 - \tfrac{1}{\gamma}\langle (\widehat{\mathcal{L}}_2 + \widehat{\mathcal{L}}_2^T)
		\mathbf{w},\mathbf{w}\rangle + \tfrac{1}{\gamma}^2\|\widehat{\mathcal{L}}_2\mathbf{w}\|^2}.
\end{align*}
%
Note that by assumption, $\gamma \geq \eta$, which implies
$0< \tfrac{\eta}{\gamma^2} \leq \tfrac{\eta^2}{\gamma^2}  < 1$, and 
$|1 - 2\tfrac{\gamma-\eta}{\gamma}| < 1$. In addition, by \Cref{ass:fov} $W(\widehat{\mathcal{L}}_2)\leq 0$,
which implies $-\langle (\widehat{\mathcal{L}}_2+\widehat{\mathcal{L}}_2^T)\mathbf{w},\mathbf{w}\rangle \geq 0$
\cite{gustafson1997numerical,mees1979domains}.
It follows that all terms in the numerator and denominator are positive, and
%
\begin{align} \label{eq:P1}
\left\| I - \tfrac{\gamma-\eta}
	{\gamma}(I - \tfrac{1}{\gamma}\widehat{\mathcal{L}}_2)^{-1}\right\|^2
& < \sup_{\mathbf{w}\neq\mathbf{0}} \frac{\|\mathbf{w}\|^2
	- \tfrac{1}{\gamma}\langle (\widehat{\mathcal{L}}_2 + \widehat{\mathcal{L}}_2^T)
		\mathbf{w},\mathbf{w}\rangle + \tfrac{1}{\gamma}^2\|\widehat{\mathcal{L}}_2\mathbf{w}\|^2}
	{\|\mathbf{w}\|^2 - \tfrac{1}{\gamma}\langle (\widehat{\mathcal{L}}_2 + \widehat{\mathcal{L}}^T)
		\mathbf{w},\mathbf{w}\rangle + \tfrac{1}{\gamma}^2\|\widehat{\mathcal{L}}\mathbf{w}\|^2} 
= 1.
\end{align}
%
Combining \eqref{eq:Pgn} and \eqref{eq:P1} yields
%
\begin{align}\label{eq:Pgamma_gen}
\|\mathcal{P}_\gamma\| \leq 1 + \frac{\beta^2}{\gamma\eta}.
\end{align}

Now consider bounding $\|\mathcal{P}_\gamma^{-1}\|$ from above. Let $s_{\max}(A)$
and $s_{\min}(A)$ denote the maximum and minimum singular value of matrix $A$,
respectively, and recall
%
\begin{align}\label{eq:sing_vals}
\|\mathcal{P}_\gamma^{-1}\| = s_{\max}(\mathcal{P}_\gamma^{-1})
	& = \frac{1}{s_{\min}(\mathcal{P}_\gamma)}, \hspace{5ex}\textnormal{where}\hspace{2ex}
s_{\min}(\mathcal{P}_\gamma) =
	\min_{\mathbf{v}\neq\mathbf{0}} \frac{\|\mathcal{P}_\gamma\mathbf{v}\|}{\|\mathbf{v}\|}.
\end{align}
%
Thus, consider the minimum singular value of $P_\gamma$. Letting $\mathbf{v} :=
(I - \tfrac{1}{\gamma}\widehat{\mathcal{L}}_2)(I - \tfrac{1}{\eta}\widehat{\mathcal{L}}_1)\mathbf{w}$,
%
\begin{align}\nonumber
s_{\min}(\mathcal{P}_\gamma)^2 & = \min_{\mathbf{v}\neq\mathbf{0}}
	\frac{\left\| \left[I - \frac{\gamma - \eta}{\gamma}
	( I- \tfrac{1}{\gamma}\widehat{\mathcal{L}}_2)^{-1} + 
	\frac{\beta^2}{\gamma\eta}( I- \tfrac{1}{\eta}\widehat{\mathcal{L}}_1)^{-1}
	( I- \tfrac{1}{\gamma}\widehat{\mathcal{L}}_2)^{-1}\right]\mathbf{v} \right\|^2}
	{\|\mathbf{v}\|^2} \\
& = \min_{\mathbf{w}\neq\mathbf{0}}
	\frac{\left\| \left[(I - \tfrac{1}{\gamma}\widehat{\mathcal{L}}_2)(I - \tfrac{1}{\eta}\widehat{\mathcal{L}}_1)
		- \frac{\gamma-\eta}{\gamma}(I - \tfrac{1}{\eta} \widehat{\mathcal{L}}_1) +
		\frac{\beta^2}{\gamma\eta} I\right]\mathbf{w} \right\|^2}
	{\|(I - \tfrac{1}{\gamma}\widehat{\mathcal{L}}_2)(I - \tfrac{1}{\eta}\widehat{\mathcal{L}}_1)\mathbf{w}\|^2} \nonumber\\
& = \min_{\mathbf{w}\neq\mathbf{0}}
	\frac{\left\| \left[(I - \tfrac{1}{\gamma}\widehat{\mathcal{L}}_2)(I - \tfrac{1}{\eta}\widehat{\mathcal{L}}_1)
		+ \frac{\gamma-\eta}{\gamma\eta}\widehat{\mathcal{L}}_1 +
		\frac{\beta^2+\eta^2 - \gamma\eta}{\gamma\eta} I\right]\mathbf{w} \right\|^2}
	{\|(I - \tfrac{1}{\gamma}\widehat{\mathcal{L}}_2)(I - \tfrac{1}{\eta}\widehat{\mathcal{L}}_1)\mathbf{w}\|^2}.
	\nonumber
\end{align}
%
Let us make the strategic choice of picking $\gamma$ such that the identity perturbation
$\tfrac{\beta^2+\eta^2 - \gamma\eta}{\gamma\eta} I = \mathbf{0}$, given by $\gamma_*
:= \tfrac{\eta^2+\beta^2}{\eta}$. Note, this is exactly the optimal $\gamma$ derived 
in \Cref{sec:theory:spd}. Expanding, we have
%
{\small
\begin{align}
& \hspace{-5ex}
s_{\min}(\mathcal{P}_\gamma)^2 = 
	\min_{\mathbf{w}\neq\mathbf{0}}
	\frac{\left\| \left[(I - \tfrac{1}{\gamma}\widehat{\mathcal{L}}_2)(I - \tfrac{1}{\eta}\widehat{\mathcal{L}}_1)
		+ \frac{\beta^2}{\eta(\eta^2+\beta^2)}\widehat{\mathcal{L}}_1\right]\mathbf{w} \right\|^2}
	{\|(I - \tfrac{1}{\gamma}\widehat{\mathcal{L}}_2)(I - \tfrac{1}{\eta}\widehat{\mathcal{L}}_1)\mathbf{w}\|^2} \nonumber\\
& = \min_{\mathbf{w}\neq\mathbf{0}} 1 +
	\frac{\beta^2}{\eta(\eta^2+\beta^2)}\cdot 
	\frac{\frac{\beta^2}{\eta(\eta^2+\beta^2)}\left\|\widehat{\mathcal{L}}_1\mathbf{w} \right\|^2
		+ 2\left\langle(I - \tfrac{1}{\gamma}\widehat{\mathcal{L}}_2)(I - \tfrac{1}{\eta}\widehat{\mathcal{L}}_1)\mathbf{w},
		\widehat{\mathcal{L}}_1\mathbf{w} \right\rangle}
	{\|(I - \tfrac{1}{\gamma}\widehat{\mathcal{L}}_2)(I - \tfrac{1}{\eta}\widehat{\mathcal{L}}_1)\mathbf{w}\|^2} \nonumber\\
& = 1 - \frac{\beta^2}{\eta^2+\beta^2} \cdot\max_{\mathbf{w}\neq\mathbf{0}}
	\frac{(-\tfrac{2}{\eta})\left\langle(I - \tfrac{1}{\gamma}\widehat{\mathcal{L}}_2)(I -
		\tfrac{1}{\eta}\widehat{\mathcal{L}}_1)\mathbf{w},
		\widehat{\mathcal{L}}_1\mathbf{w} \right\rangle- 
		\frac{\beta^2}{\eta^2(\eta^2+\beta^2)}\left\|\widehat{\mathcal{L}}_1\mathbf{w} \right\|^2}
	{\|(I - \tfrac{1}{\gamma}\widehat{\mathcal{L}}_2)(I - \tfrac{1}{\eta}\widehat{\mathcal{L}}_1)\mathbf{w}\|^2}.
	\label{eq:gen_smin}
\end{align}
}
%
Expanding the numerator term, we have
{\small
\begin{align}\nonumber
& (-\tfrac{2}{\eta})\left\langle(I - \tfrac{1}{\gamma}\widehat{\mathcal{L}}_2)(I - \tfrac{1}{\eta}\widehat{\mathcal{L}}_1)\mathbf{w},
		\widehat{\mathcal{L}}_1\mathbf{w} \right\rangle- 
		\frac{\beta^2}{\eta^2(\eta^2+\beta^2)}\left\|\widehat{\mathcal{L}}_1\mathbf{w} \right\|^2 \\
% & = \left(\frac{2}{\eta^2} - \frac{\beta^2}{\eta^2(\eta^2+\beta^2)}\right)
% 			\left\|\widehat{\mathcal{L}}_1\mathbf{w} \right\|^2
% 		- \frac{2}{\eta}\langle\mathbf{w},\widehat{\mathcal{L}}_1\mathbf{w}\rangle
% 		- \frac{2}{\gamma\eta^2}\langle\widehat{\mathcal{L}}_2(\widehat{\mathcal{L}}_1\mathbf{w}),\widehat{\mathcal{L}}_1\mathbf{w}\rangle
% 		+ \frac{2}{\gamma\eta}\langle\widehat{\mathcal{L}}_1\mathbf{w},\widehat{\mathcal{L}}_2\mathbf{w}\rangle \nonumber\\
& = \left(\frac{1}{\eta^2} + \frac{1}{\eta^2+\beta^2}\right)
			\left\|\widehat{\mathcal{L}}_1\mathbf{w} \right\|^2
		- \frac{2}{\eta}\langle\mathbf{w},\widehat{\mathcal{L}}_1\mathbf{w}\rangle
		- \frac{2}{\gamma\eta^2}\langle\widehat{\mathcal{L}}_2(\widehat{\mathcal{L}}_1\mathbf{w}),\widehat{\mathcal{L}}_1\mathbf{w}\rangle
		+ \frac{2}{\gamma\eta}\langle\widehat{\mathcal{L}}_1\mathbf{w},\widehat{\mathcal{L}}_2\mathbf{w}\rangle.
		\label{eq:num_gen}
\end{align}
}
%
Now consider the denominator:
%
\begin{align}
& \hspace{-10ex}
\left\|(I - \tfrac{1}{\gamma}\widehat{\mathcal{L}}_2)(I - \tfrac{1}{\eta}\widehat{\mathcal{L}}_1)\mathbf{w}\right\|^2
= \left\|\left(I + \tfrac{1}{\gamma\eta}\widehat{\mathcal{L}}_2\widehat{\mathcal{L}}_1\right)\mathbf{w} - 
	\left(\tfrac{1}{\gamma}\widehat{\mathcal{L}}_2 + \tfrac{1}{\eta}\widehat{\mathcal{L}}_1\right)\mathbf{w}\right\|^2 \nonumber\\
& = \left\|\left(I + \tfrac{1}{\gamma\eta}\widehat{\mathcal{L}}_2\widehat{\mathcal{L}}_1\right)\mathbf{w}\right\|^2
	+ \frac{1}{\gamma^2}\|\widehat{\mathcal{L}}_2\mathbf{w}\|^2
	+ \frac{1}{\eta^2}\|\widehat{\mathcal{L}}_1\mathbf{w}\|^2
	+ \frac{2}{\gamma\eta}\langle\widehat{\mathcal{L}}_1\mathbf{w},\widehat{\mathcal{L}}_2\mathbf{w}\rangle
	\nonumber\\ & \hspace{5ex}
	- \frac{2}{\gamma}\Big\langle \left(I + \tfrac{1}{\gamma\eta}\widehat{\mathcal{L}}_2\widehat{\mathcal{L}}_1\right)\mathbf{w},
		\widehat{\mathcal{L}}_2\mathbf{w}\Big\rangle
	- \frac{2}{\eta}\Big\langle \left(I + \tfrac{1}{\gamma\eta}\widehat{\mathcal{L}}_2\widehat{\mathcal{L}}_1\right)\mathbf{w},
		\widehat{\mathcal{L}}_1\mathbf{w}\Big\rangle \nonumber\\
& \geq \left\|\left(I + \tfrac{1}{\gamma\eta}\widehat{\mathcal{L}}_2\widehat{\mathcal{L}}_1\right)\mathbf{w}\right\|^2
	+ \frac{1}{\gamma^2}\|\widehat{\mathcal{L}}_2\mathbf{w}\|^2
	+ \frac{1}{\eta^2}\|\widehat{\mathcal{L}}_1\mathbf{w}\|^2
	+ \frac{2}{\gamma\eta}\langle\widehat{\mathcal{L}}_1\mathbf{w},\widehat{\mathcal{L}}_2\mathbf{w}\rangle
	\nonumber\\ & \hspace{5ex}
	- \frac{2}{\gamma}\left\| \left(I + \tfrac{1}{\gamma\eta}\widehat{\mathcal{L}}_2\widehat{\mathcal{L}}_1\right)\mathbf{w}\right\|
		\left\|\widehat{\mathcal{L}}_2\mathbf{w}\right\|
	- \frac{2}{\eta}\Big\langle \left(I + \tfrac{1}{\gamma\eta}\widehat{\mathcal{L}}_2\widehat{\mathcal{L}}_1\right)\mathbf{w},
		\widehat{\mathcal{L}}_1\mathbf{w}\Big\rangle \nonumber\\
& = \left( \left\|\left(I + \tfrac{1}{\gamma\eta}\widehat{\mathcal{L}}_2\widehat{\mathcal{L}}_1\right)\mathbf{w}\right\|
		- \frac{1}{\gamma}\|\widehat{\mathcal{L}}_2\mathbf{w}\|\right)^2
	+ \frac{1}{\eta^2}\|\widehat{\mathcal{L}}_1\mathbf{w}\|^2
	\nonumber\\ & \hspace{5ex}
	- \frac{2}{\eta}\Big\langle \left(I + \tfrac{1}{\gamma\eta}\widehat{\mathcal{L}}_2\widehat{\mathcal{L}}_1\right)\mathbf{w},
		\widehat{\mathcal{L}}_1\mathbf{w}\Big\rangle
	+ \frac{2}{\gamma\eta}\langle\widehat{\mathcal{L}}_1\mathbf{w},\widehat{\mathcal{L}}_2\mathbf{w}\rangle \nonumber\\
& \geq \frac{1}{\eta^2}\|\widehat{\mathcal{L}}_1\mathbf{w}\|^2
	- \frac{2}{\eta}\langle \mathbf{w}, \widehat{\mathcal{L}}_1\mathbf{w}\rangle 
	- \frac{2}{\gamma\eta^2}\langle \widehat{\mathcal{L}}_2(\widehat{\mathcal{L}}_1\mathbf{w}),
		\widehat{\mathcal{L}}_1\mathbf{w}\rangle
	+ \frac{2}{\gamma\eta}\langle\widehat{\mathcal{L}}_1\mathbf{w},\widehat{\mathcal{L}}_2\mathbf{w}\rangle.
		\label{eq:den_gen}
\end{align}
%

By assumption, $\langle\widehat{\mathcal{L}}_1\mathbf{w},\widehat{\mathcal{L}}_2\mathbf{w}\rangle\geq 0$,
and thus all terms in \eqref{eq:num_gen} and \eqref{eq:den_gen} are non-negative.
Returning to the minimum singular value defined in \eqref{eq:gen_smin} and plugging in
the numerator \eqref{eq:num_gen} and denominator bounds \eqref{eq:den_gen} yields
an upper bound on the maximum over $\mathbf{w}$,
%
{\small
\begin{align}\label{eq:max_bound0}
&\max_{\mathbf{w}\neq\mathbf{0}}
	\frac{(-\tfrac{2}{\eta})\left\langle(I - \tfrac{1}{\gamma}\widehat{\mathcal{L}}_2)(I -
		\tfrac{1}{\eta}\widehat{\mathcal{L}}_1)\mathbf{w},
		\widehat{\mathcal{L}}_1\mathbf{w} \right\rangle- 
		\frac{\beta^2}{\eta^2(\eta^2+\beta^2)}\left\|\widehat{\mathcal{L}}_1\mathbf{w} \right\|^2}
	{\|(I - \tfrac{1}{\gamma}\widehat{\mathcal{L}}_2)(I - \tfrac{1}{\eta}\widehat{\mathcal{L}}_1)\mathbf{w}\|^2} \\
& \leq \max_{\mathbf{w}\neq\mathbf{0}}\frac{\frac{2\eta^2+\beta^2}{\eta^2(\eta^2+\beta^2)}
			\left\|\widehat{\mathcal{L}}_1\mathbf{w} \right\|^2
		- \frac{2}{\eta}\langle\mathbf{w},\widehat{\mathcal{L}}_1\mathbf{w}\rangle
		- \frac{2}{\gamma\eta^2}\langle\widehat{\mathcal{L}}_2(\widehat{\mathcal{L}}_1\mathbf{w}),
			\widehat{\mathcal{L}}_1\mathbf{w}\rangle
		+ \frac{2}{\gamma\eta}\langle\widehat{\mathcal{L}}_1\mathbf{w},\widehat{\mathcal{L}}_2\mathbf{w}\rangle}
	{\frac{1}{\eta^2}\|\widehat{\mathcal{L}}_1\mathbf{w}\|^2
	- \frac{2}{\eta}\langle \mathbf{w}, \widehat{\mathcal{L}}_1\mathbf{w}\rangle 
	- \frac{2}{\gamma\eta^2}\langle \widehat{\mathcal{L}}_2(\widehat{\mathcal{L}}_1\mathbf{w}),
		\widehat{\mathcal{L}}_1\mathbf{w}\rangle
	+ \frac{2}{\gamma\eta}\langle\widehat{\mathcal{L}}_1\mathbf{w},\widehat{\mathcal{L}}_2\mathbf{w}\rangle} \nonumber\\
& \leq \frac{\frac{2\eta^2+\beta^2}{\eta^2(\eta^2+\beta^2)}}{\frac{1}{\eta^2}}.\label{eq:max_bound}
% & = \frac{2\eta^2 + \beta^2}{\eta^2+\beta^2}.
\end{align}
}
%
Simplifying and plugging in to \eqref{eq:gen_smin} yields
%
\begin{align}\label{eq:smin_bound}
s_{\min}(\mathcal{P}_{\gamma_*})^2 &\geq 1 - \frac{\beta^2}{\eta^2+\beta^2} \cdot
	\frac{2\eta^2 + \beta^2}{\eta^2+\beta^2}
= \frac{\eta^4}{(\eta^2+\beta^2)^2}.
\end{align}
%
Applying \eqref{eq:sing_vals} to \eqref{eq:smin_bound} and
combining with \eqref{eq:Pgamma_gen} yields
%
\begin{align}
\textnormal{cond}(\mathcal{P}_{\gamma_*}) = \|\mathcal{P}_{\gamma_*}\|\|\mathcal{P}_{\gamma_*}^{-1}\|
	\leq \left(1+\frac{\eta^2}{\eta^2+\beta^2}\right)\frac{\eta^2+\beta^2}{\eta^2}
	= 2+\frac{\beta^2}{\eta^2},
\end{align}
%
which completes the proof.
\end{proof}
%

%
\begin{remark}[$\widehat{\mathcal{L}}_1 = \widehat{\mathcal{L}}_2$]
If $\widehat{\mathcal{L}}_1 = \widehat{\mathcal{L}}_2$, similar derivations as in
\Cref{th:cond} yield a tighter condition number
\begin{align*}
\textnormal{cond}(\mathcal{P}_{\gamma_*}) \leq
	2+\frac{\beta^2}{2\eta^2} - \frac{\beta^2}{2\eta^2 + 2\beta^2}.
\end{align*}
This still has dependence on $\eta$ and $\beta$, but grows at about half the rate
of the bound derived for $\widehat{\mathcal{L}}_1 \neq \widehat{\mathcal{L}}_2$.
It is worth pointing out that some of the PRESB methods \todo{cite} may also be
effective for $\widehat{\mathcal{L}}_1 = \widehat{\mathcal{L}}_2$ (or potentially
more effective than the block preconditioning considered here). We do not consider
such methods because the $2\times 2$ Schur complement approach has proven effective
in practice and, moreover, PRESB methods do not naturally generalize to
$\widehat{\mathcal{L}}_1 \neq \widehat{\mathcal{L}}_2$, which is one of the
focuses of this paper.
\end{remark}
%

%
\begin{remark}[(Lack of) tightness of bounds]
Although \Cref{th:cond} provides condition number bounds $\sim\mathcal{O}(1)$, they
are likely not particularly tight. There are a number of approximations made in the
proof that are difficult to quantify or tighten, but likely pessimistic for realistic
problems. For example,
the maximum in \eqref{eq:max_bound} is taken as the largest ratio of matching numerator
and denominator terms. However, three of the four terms have a unity ratio, and only
the $\|\widehat{\mathcal{L}}_1\mathbf{w}\|^2$ term is non-unity. In practice, these
other terms will often be of non-trivial size and the maximum in \eqref{eq:max_bound0}
would be closer to one than the bound in \eqref{eq:max_bound}. If \eqref{eq:max_bound0}
is bounded above by one, $\textnormal{cond}(\mathcal{P}_{\gamma_*})$ is bounded
by a function that scales like $\sqrt{\beta^2/\eta^2}$, and $\approx 10$ for
$\beta^2/\eta^2 = 100$. Other approximations that are likely pessimistic in practice
are made in bounding the denominator \eqref{eq:den_gen}.
\end{remark}
%


% ------------------------------------------------------------------------------------- %
% ------------------------------------------------------------------------------------- %
% ------------------------------------------------------------------------------------- %
\section{Nonlinear iterations}\label{sec:nonlinear}

% In \todo{cite}, an IRK preconditioning method is developed that can be applied on the
% solution level, rather than solving for all stages and updating the solution
% by summing over stages.
% This is beneficial from a memory perspective, being able to achieve very
% high order accuracy while only storing the solution and an auxilliary vector, and
% also allows for the use of CG and MINRES when applicable to the spatial operator.

% The time-dependent and nonlinear case is more complicated. Consider
% the simplest cast of a time-independent nonlinear problem and simplified Newton method,
% which only applies the Jacobian based on the current solution. Each
% nonlinear iteration requires the action of the nonlinear operator to compute a residual,
% and this action is implicitly defined by individual stage vectors. The linear algorithm
% developed in \Cref{sec:solve} solves for the summation over stage vectors, and individual
% stages cannot be extracted. Without each stage vector, the action of the nonlinear 
% operator cannot be computed, and even the simplified Newton method cannot be applied.
% Similar difficulties apply for full Newton and Picard iterations for operators with
% and without time-dependent differential components. Reformulating the algorithm
% from \Cref{sec:solve} to store stage vectors also ends up being impractical --
% applying det$\mathcal{M}_s^{-1}$ requires the solution of $s$ linear systems. To store
% each stage vector requires applying det$\mathcal{M}_s^{-1}$ to each one. This
% requires $s^2$ linear solves, which is too expensive to be appealling in practice.


% ------------------------------------------------------------------------------------- %
% ------------------------------------------------------------------------------------- %
\subsection{Simplified Newton}\label{sec:nonlinear:simp}

Suppose $\mathcal{L}_i = \mathcal{L}_j$ for all $i,j$ (as in a simplified Newton method).
Then, the linear system for stage vectors \eqref{eq:keq} can be written in condensed
Kronecker product notation
%
\begin{align}\label{eq:keq2}
\left( A_0^{-1}\otimes I - I\otimes\widehat{\mathcal{L}}\right)
	(A_0\otimes I) \mathbf{k} & = \mathbf{f}.
\end{align}
%
Now, let $A_0^{-1} = Q_0R_0Q_0^T$ be the real Schur decomposition of $A_0^{-1}$, where
$Q_0$ is real-valued and orthogonal, and $R_0$ is a block
upper triangular matrix, where each block corresponds to an eigenvalue (pair) of
$A_0^{-1}$. Real-valued eigenvalues have block size one, and complex eigenvalues
$\eta\pm i\beta$ are in $2\times 2$ blocks,
$\begin{bmatrix} \eta & \phi \\-\beta^2/\phi & \eta\end{bmatrix}$, for some
constant $\phi$.
Pulling out a $Q_0\otimes I$ and $Q_0^T\otimes I$ from the left and right of
\eqref{eq:keq2} yields the equivalent linear system
%
\begin{align}\label{eq:keq3}
\left( R_0\otimes I - I \otimes \widehat{\mathcal{L}}\right)
	(R_0^{-1}Q_0^T\otimes I) \mathbf{k} & = (Q_0^T\otimes I)\mathbf{f}.
\end{align}
%
The left-most matrix is now block upper triangular, which can be solved
using block backward substitution, and requires inverting each diagonal block.
Diagonal blocks corresponding to real-valued eigenvalues $\eta$ take the form
$(\eta I - \widehat{\mathcal{L}})$, and are amenable to standard preconditioning
techniques. $2\times 2$ diagonal blocks corresponding to complex eigenvalues
take the form
%
\begin{align}\label{eq:block}
\begin{bmatrix} \eta I - \widehat{\mathcal{L}} & \phi I\\
-\frac{\beta^2}{\phi} I & \eta I - \widehat{\mathcal{L}}\end{bmatrix}.
\end{align}
%

Convergence of GMRES with a block lower triangular preconditioner as in
\eqref{eq:Lprec} is then based on the preconditioned Schur complement 
problem. \Cref{th:eig} (symmetric semi-definite $\widehat{\mathcal{L}}$)
and \Cref{th:cond} (general $\widehat{\mathcal{L}}$) prove that defining
$\widehat{S}:= \gamma_*I -  \widehat{\mathcal{L}}$ yields a preconditioned
Schur complement with $\mathcal{O}(1)$ conditioning. In practice, we typically
do not want to apply $(\eta I - \widehat{\mathcal{L}})^{-1}$ or
$\widehat{S}^{-1}$ exactly for each iteration of the preconditioner
\eqref{eq:Lprec}. It is well-known in the block-preconditioning community
that a few iterations of an effective preconditioner, such as multigrid,
to represent the inverse of diagonal blocks in \eqref{eq:Lprec} typically
yields convergence on the larger $2\times 2$ operator just as fast as if
performing direct solves, at a fraction of the cost. Thus, we propose
a block-triangular preconditioner similar to \eqref{eq:Lprec}, but which
only applies some approximation to the diagonal block inverses,
$(\eta I - \widehat{\mathcal{L}})^{-1}$ and
$\widehat{S}^{-1} := (\gamma_* I - \widehat{\mathcal{L}})^{-1}$. 

%
\begin{remark}[Real Schur decomposition]
It should be pointed out that a real Schur decomposition is by no means
unique to Runge-Kutta literature. Many papers have considered similar
approaches as above, particularly for systems of ODEs (rather than PDEs)
\todo{cite}.
The key contribution here for the simplified Newton setting is proving a
robust and general way to precondition the resulting operators in the
context of numerical PDEs (see \Cref{sec:theory}). Moreover, the real
Schur decomposition applied to the simplified Newton setting after
pulling out an $A_0^{-1}\otimes I$ provides the key motivation for
the development of more general nonlinear iterations introduced in 
the following section.
\end{remark}


%
\begin{comment}
\begin{remark}[Closed form Inverse]
One might note that the two Schur complements of \eqref{eq:block} are the same, and
\begin{align*}
S^{-1} = \mathcal{Q}_\eta^{-1}(\eta I - \widehat{\mathcal{L}}),
\end{align*}
with $\mathcal{Q}_\eta$ as in \eqref{eq:imag1}. Using this, one can get a simple
closed form for the inverse of \eqref{eq:imag1} based on $\mathcal{Q}_\eta^{-1}$.
However, consistent with the algorithm developed in \Cref{sec:solve}, applying the
$2\times 2$ inverse requires applying $\mathcal{Q}_\eta^{-1}$ to both stage vectors,
which doubles the number of linear solves. 
\end{remark}
%


% See "Parallel iterative linear solvers for multistep Runge-Kutta methods"
%
%
%
\begin{remark}[A similar approach for $A_0$]
This is not the first work to consider a real Schur decomposition of $A_0$
\todo{cite?}. A similar approach can be applied to the traditional Kronkecker
product form \eqref{eq:kron1}, now transforming $A_0$ (instead of $A_0^{-1}$)
using the real Schur decomposition. Solving for all stage vectors can then 
be achieved by solving a similar block upper triangular matrix as in
\eqref{eq:keq3}, but now the $2\times 2$ block linear systems correspond
to complex eigenvalues of $A_0$, say $\mu + i\zeta$, and take the form 
\begin{align*}
\begin{bmatrix} I - \mu\widehat{\mathcal{L}} & \phi\widehat{\mathcal{L}} \\
	-\frac{\zeta^2}{\phi}\widehat{\mathcal{L}} & I - \mu\widehat{\mathcal{L}} \end{bmatrix}.
\end{align*}

An analogous approach as used with the $A_0^{-1}$ transformation
can be applied here. If we precondition the Schur complement
$S_0 = (I - \mu \widehat{\mathcal{L}}) +
\zeta^2(I - \mu \widehat{\mathcal{L}})^{-1}\widehat{\mathcal{L}}^2$ with the
term associated with the real-valued eigenvalue, $(I - \mu \widehat{\mathcal{L}})$,
the preconditioned Schur complement is given by
%
\begin{align*}
(I - \mu \widehat{\mathcal{L}})^{-1}S_0
	& = I -  \frac{\zeta^2}{\mu^2}(I - \tfrac{1}{\mu} \widehat{\mathcal{L}}^{-1})^{-2}.
\end{align*}
%
Note that this closely resembles the preconditioned operator from \eqref{eq:prec1}
and \eqref{eq:prec2}, but here with a $\widehat{\mathcal{L}}^{-1}$ inside of the
inverse term, as opposed to a $\widehat{\mathcal{L}}$ in \eqref{eq:prec1}
and \eqref{eq:prec2}. In \Cref{th:fov}, we assumed that $\widehat{\mathcal{L}}$
is negative in a field of values sense, equivalent to saying that
$(\widehat{\mathcal{L}} + \widehat{\mathcal{L}}^T)$ is negative semi-definite.
However, it holds that if $(\widehat{\mathcal{L}} + \widehat{\mathcal{L}}^T)$ is
negative semi-definite, then $(\widehat{\mathcal{L}}^{-1} +
\widehat{\mathcal{L}}^{-T})$ is also negative semi-definite, which yields the
following corollary.
\end{remark}
%

%
\begin{corollary}[Preconditioned field of values]\label{cor:fov}
Assume that $\mu > 0$ and the symmetric part of $\widehat{\mathcal{L}}^{-1}$
satisfies $(\widehat{\mathcal{L}}^{-1}+\widehat{\mathcal{L}}^{-T}) \leq 0$. Let
$\mathcal{P}_\mu$ denote the preconditioned operator, where
$(I - \mu \widehat{\mathcal{L}}) +
	\zeta^2(I - \mu \widehat{\mathcal{L}})^{-1}\widehat{\mathcal{L}}^2$ is
preconditioned with $(\mu I - \widehat{\mathcal{L}})^{-1}$. Then 
$W(\mathcal{P}_\mu)$ is contained within a region analogous to $\Omega$ 
in \Cref{fig:bound}, but with radius $\zeta^2/\mu^2$ (based on the eigenvalues
of $A_0$) rather than $\beta^2/\eta^2$. 
\end{corollary}
\begin{proof}
The proof follows from the above discussion and analagous derivations to the
proof of \Cref{th:fov}.
\end{proof}
%

\todo{rewrite, comment on $\beta^2/\eta^2$ and $\zeta^2/\mu^2$ being the same
for all Gauss and RadauIIA methods.}
\Cref{cor:fov} suggests the $A_0^{-1}$ transformation is not necessary -- one
can apply the real Schur decomposition to the traditional Kronecker product form 
\eqref{eq:kron1} and use a similar block preconditioning as introduced in 
\eqref{eq:Lprec} and \eqref{eq:prec2} to solve the $2\times 2$ blocks associated
with complex eigenvalues, with similarly nice theoretical bounds on the
conditioning of the preconditioned operator. Here, we use the $A_0^{-1}$
approach as it \textit{is} necessary for the linear algorithm developed in
\Cref{sec:solve}, and also more naturally suggests an extension to the
general nonlinear setting (for example, full Newton as opposed to simplified
Newton) in the following section.
\end{comment}


% ------------------------------------------------------------------------------------- %
% ------------------------------------------------------------------------------------- %
\subsection{General nonlinear iterations}\label{sec:nonlinear:gen}

We start by noting that most nonlinear iterations, including Newton, Picard, and
other fixed-point iterations, can all be expressed as linearly preconditioned
nonlinear Richardson iterations. For nonlinear functional
$\mathcal{F}(\mathbf{x}) = 0$, such an iteration takes the form
%
\begin{align}\label{eq:non_rich}
\mathbf{x}_{k+1} = \mathbf{x}_k + \mathcal{P}^{-1}\mathcal{F}(\mathbf{x}_k).
\end{align}
%
For preconditioner $\mathcal{P} := -J[\mathbf{x}_k]$ given by the (negative)
Jacobian of $\mathcal{F}(\mathbf{x})$ evaluated at $\mathbf{x}_k$, \eqref{eq:non_rich}
yields a Newton iteration. For $\mathcal{P}$ given by a zero-th order linearization
of $\mathcal{F}(\mathbf{x})$ (the nonlinear operator evaluated at $\mathbf{x}_k$),
\eqref{eq:non_rich} yields a Picard iteration. In general, thinking of nonlinear
iterations as linear preconditioners for nonlinear Richardson iterations
\eqref{eq:non_rich} naturally allows for various levels of approximation,
which is the focus of this section. 

Now let us return to \eqref{eq:keq} for $\mathcal{L}_i\neq\mathcal{L}_j$, but
extract the real Schur decomposition as in \Cref{sec:nonlinear:gen}. Continuing
with the simplified representation $\widehat{\mathcal{L}}_i := \delta t M^{-1}\mathcal{L}_i$,
this yields the linear system
%
\begin{align}\label{eq:keq4}
\left( R_0\otimes I - (Q_0^T\otimes I) \begin{bmatrix}
	\widehat{\mathcal{L}}_1  & \\ & \ddots \\ && \widehat{\mathcal{L}}_s\end{bmatrix}
	(Q_0\otimes I)\right) (R_0^{-1}Q_0^T\otimes I) \mathbf{k}\
= (Q_0^T\otimes I)\mathbf{f}.
\end{align}
%
Picard and Newton iterations both require the solution of such a system each
iteration (see $\mathcal{P}^{-1}$ in \eqref{eq:non_rich}). Here we propose 
approximations to the solution of \eqref{eq:keq4} that are (i) solvable using
techniques similar to \Cref{sec:nonlinear:simp}, and (ii) yield nonlinear 
convergence close to a true Newton or Picard iteration. In principle, these
approximations can also be iterated to convergence in the linear sense, yielding
a precise Newton or Picard iteration, but here we opt to resolve the approximations
(i.e., converge) in the nonlinear iteration. Similar to inexact Newton methods,
such an approach is often more efficient in practice than computing, e.g., an exact
Jacobian. 

% Note, it is important to express Picard and fixed-point iterations in the 
% form of a Newton-like method to use the preconditioning techniques developed
% in this paper! By expressing it as a Newton-like method, we are approximating
% the Jacobian and will still converge to the solution of the nonlinear functional
% $\mathcal{F} = 0$ when adding levels of approximation as in \Cref{sec:nonlinear:gen}.
% A traditional Picard iteration relies on converging exactly to the fixed-point
% $\mathcal{G}(\mathbf{k}) = \mathbf{k}$ via iterates $\mathbf{k}^{k+1} =
% \mathcal{G}(\mathbf{k}^k)$. Within this iteration there is delicate cancellation
% that result from solving directly for $\mathbf{k}^{k+1}$ rather than a correction,
% and approximating $\mathcal{G}$ would result in convergence to a different problem.

To develop effective approximations, we are particularly interested in the operator
%
\begin{align}\label{eq:Q0approx}
\widehat{P} \coloneqq (Q_0^T\otimes I) \begin{bmatrix}
	\widehat{\mathcal{L}}_1  & \\ & \ddots \\ && \widehat{\mathcal{L}}_s\end{bmatrix}
	(Q_0\otimes I)
= \begin{bmatrix}
	\bm{c}^\top_{1,1} \bm{{\cal L}}' & \cdots & \bm{c}^\top_{1,s} \bm{{\cal L}}' \\
	\vdots & & \vdots \\
	\bm{c}^\top_{s,1} \bm{{\cal L}}' & \cdots & \bm{c}^\top_{s,s} \bm{{\cal L}}'
	\end{bmatrix},
\end{align}
%
where $\bm{c}^\top_{k,\ell} = \Big((Q_0^\top)_{k ,1} (Q_0)_{1, \ell},
\ldots, (Q_0^\top)_{k, s} (Q_0)_{s, \ell} \Big) \in \mathbb{R}^s$ is a scalar row
vector, $\bm{{\cal L}}' = (\widehat{\mathcal{L}}_1; \ldots; \widehat{\mathcal{L}}_s)$
is a block column vector of the linearized operators, and
%
\begin{align*}
\bm{c}^\top_{k,\ell} \bm{{\cal L}}' = \sum \limits_{i = 1}^s (\bm{c}_{k, \ell})_i
	\widehat{\mathcal{L}}_i.
\end{align*}
%
For $\widehat{\mathcal{L}}_i = \widehat{\mathcal{L}}_j$, \eqref{eq:Q0approx}
is block diagonal, given by $I\otimes\widehat{\mathcal{L}}$. For
$\widehat{\mathcal{L}}_i \neq \widehat{\mathcal{L}}_j$, note that
$\sum_{i=1}^s (\bm{c}_{k, \ell})_i = 1$, while for off-diagonal blocks
$\sum_{i=1}^s (\bm{c}_{k, \ell})_i = 0$.
Due to the off-diagonal zero sums, here we claim that \eqref{eq:Q0approx}
can be well-approximated by some block-diagonal or block upper triangular
approximation, which can then be easily inverted using block backward
substitution.

As an example, consider \eqref{eq:Q0approx} for the two-stage Gauss and
RadauIIA methods in bracket notation, where, e.g., $\{a_1,a_2,a_3\}\mapsto 
a_1\widehat{\mathcal{L}}_1 + a_2\widehat{\mathcal{L}}_2 + a_3\widehat{\mathcal{L}}_3$:
%
\begin{align*}
\textnormal{Gauss(2):} \hspace{1ex}
	\begin{bmatrix}
	\{1,0\} & \{0,0\} \\
	 \{0,0\} & \{0,1\} \\
	\end{bmatrix},
	\hspace{3ex}
\textnormal{RadauIIA(2):} \hspace{1ex}
	\begin{bmatrix}
	\{0.985,0.014\} & \{0.121,-0.121\}\\
	\{0.121,-0.121\} & \{0.014,0.985\}\\
	\end{bmatrix}.
\end{align*}
%
Recall in the matrices above, the constants in diagonal brackets always sum to
one and off-diagonal sum to zero, wherein if $\widehat{\mathcal{L}}_i = \widehat{\mathcal{L}}_j$
for all $i,j$, each of the operators is the identity.

Note that there is no approximation in two-stage Gauss, that is, it is
straightforward to apply a true Newton or Picard iteration to two-stage
Gauss using preconditioning techniques developed in \Cref{sec:nonlinear:simp}.
For two-stage RadauIIA, we see that the diagonal blocks are almost
defined by the (linearized) operator evaluated at a single time step, which
provides a natural and simple approximation. The off-diagonal blocks are
simply a measure of commutation, $0.121(\widehat{\mathcal{L}}_1 - \widehat{\mathcal{L}}_2)$.
Such entries could be included in the preconditioning for the upper
triangular portion of the matrix (adding a few additional mat-vecs and
memory usage), or simply ignored under the assumption that $\widehat{\mathcal{L}}_1 -
\widehat{\mathcal{L}}_2$ is ``small'' in some sense. Even for reasonably stiff 
problems, the operator often does not change substantially between
two stages -- if it did, e.g., due to a shockwave, the larger time step
may not be small enough to accurately resolve the nonlinear behavior in
the first place. Similar structure as discussed for the two-stage methods
holds for other methods as well. For example, the coefficients for four-stage
Gauss are given by:
%
\begin{align*}
\textnormal{Gauss(4):}& \hspace{1ex}
\left[\begin{matrix}
\{0.002,0.014,0.012,\mathbf{0.970}\} & \{0.001,-0.011,-0.110,0.120\} \\
\{0.001,-0.011,-0.110,0.120\} & \{0.000,0.008,\mathbf{0.975},0.014\} \\
\{-0.016,-0.113,0.010,0.119\} & \{-0.007,0.085,-0.093,0.014\} \\
\{-0.045,0.041,-0.005,0.009\} & \{-0.021,-0.031,0.051,0.001\}
\end{matrix}\right.
\\&\hspace{5ex}
\left.\begin{matrix}
\{-0.016,-0.113,0.010,0.119\} & \{-0.045,0.041,-0.005,0.009\}\\
\{-0.007,0.085,-0.093,0.014\} & \{-0.021,-0.031,0.051,0.001\}\\
\{0.113,\mathbf{0.863},0.008,0.014\} & \{0.316,-0.312,-0.004,0.001\}\\
\{0.316,-0.312,-0.004,0.001\} & \{\mathbf{0.883},0.113,0.002,0.000\}\\
\end{matrix}\right].
\end{align*}
%
Analogous to the two-stage case, note that the diagonal blocks are largely
defined by $\widehat{\mathcal{L}}$ evaluated at a single time point (coefficient
shown in bold). Moreover, not only are the off-diagonal constants
a measure of commutivity, many of them are quite small in the first place;
then, regardless of commutivity, ignoring such terms in the lower-triangular
or off-diagonal blocks is a natural preconditioning. 

Motivated by the above discussion, we construct {block upper triangular}
$\widetilde{P} \approx \widehat{P}$ as Newton-like methods (or more
generally some fixed-point iteration as in \eqref{eq:non_rich}) which has
a (block) sparsity pattern contained within that of $R_0 \otimes I$. Recall
by constructing $\widetilde{P}$ to be block upper triangular, we can then
invert the resulting operator
%
\begin{align}
R_0 \otimes M - \delta t \widehat{P} \approx R_0 \otimes I -  \delta t \widetilde{P}
\end{align}
%
via block backward substitution, preconditioning each $1\times 1$
or $2\times 2$ diagonal block as in \Cref{sec:nonlinear:simp}.
In addition to the simplified Newton
method discussed in \Cref{sec:nonlinear:simp}, we propose three (successively
more accurate) approximations to \eqref{eq:Q0approx}:
\vspace{1ex}
%
\begin{enumerate}
\setlength\itemsep{0.5em}
\item[0.] \underline{Simplified Newton:} as in \Cref{sec:nonlinear:simp}, apply a
simplified Newton method by evaluating $\mathcal{L}$ at the same time point for
all stages.

\item \underline{Newton-Like(1):} Truncate $\widehat{P}$ to be block diagonal and
lump the coefficients of $\bm{c}_{ii}$ to the largest one so that each diagonal
block of $\widehat{P}$ contains only one matrix from $\bm{{\cal L}}'$.

\item \underline{Newton-Like(2):} Truncate $\widehat{P}$ to be block diagonal
(can include the off-diagonal entries in $2 \times 2$ diagonal blocks, or not)
\todo{How is this implemented?}

\item \underline{Newton-Like(3):} Truncate $\widehat{P}$ to be block upper triangular.
This option adds a number of matrix-vector products and requires storing $\mathcal{L}_i$ 
in memory for multiple $i$, but is also the best approximation to an exact Newton or
Picard iteration.

\end{enumerate}


% ------------------------------------------------------------------------------------- %
% ------------------------------------------------------------------------------------- %
% ------------------------------------------------------------------------------------- %
\section{Differential algebraic equations}\label{sec:dae}



% ------------------------------------------------------------------------------------- %
% ------------------------------------------------------------------------------------- %
% ------------------------------------------------------------------------------------- %
\section{Numerical results}\label{sec:numerics_nonlin}

% ------------------------------------------------------------------------------------- %
% ------------------------------------------------------------------------------------- %
\subsection{Nonlinear advection-diffusion}\label{sec:numerics_nonlin:advdiff}



% ------------------------------------------------------------------------------------- %
% ------------------------------------------------------------------------------------- %
\subsection{Nonlinear heat conduction}\label{sec:numerics_nonlin:heat}

Consider Example 16p in MFEM, which solves a nonlinear heat conduction equation
%
\begin{align*}
u_t  & = \nabla\cdot (\kappa + \alpha u) \nabla u.
\end{align*}
%
Let us express the diffusion as a nonlinear matrix-valued operator $K(u)$ with
potential forcing function $f(t)$. Discretizing in space we can express the semidiscrete 
system as the nonlinear set of equations
%
\begin{align*}
M\mathbf{u}_t & = K(\mathbf{u})\mathbf{u} + \mathbf{f}.
\end{align*}
%
Here we demonstrate how a typical Picard linearization can fit into the 
framework developed above. Let $\mathbf{u}_i$ denote $\mathbf{u}$ at the $i$th
time step, and consider a backward Euler step,
%
\begin{align*}
M\mathbf{u}_{i+1} & = M\mathbf{u}_i + \delta tK(\mathbf{u}_{i+1}) + \mathbf{f}.
\end{align*}
%
Rearranging yields a nonlinear system $\mathcal{F}(\mathbf{u}_{i+1}) = \mathbf{0}$,
where
%
\begin{align*}
\mathcal{F}(\mathbf{u}_{i+1}) := 
	(M - \delta t K(\mathbf{u}_{i+1}))\mathbf{u}_{i+1} - M\mathbf{u}_i - \mathbf{f}.
\end{align*}
%
We can invert the matrix-valued operator as a linear preconditioning, leading
to an equivalent nonlinear system $\mathcal{H}(\mathbf{u}_{i+1}) = \mathbf{0}$,
where
%
\begin{align*}
\mathcal{H}(\mathbf{u}_{i+1}) := 
	(M - \delta t K(\mathbf{u}_{i+1}))^{-1}(M\mathbf{u}_i + \mathbf{f}) - \mathbf{u}_{i+1}.
\end{align*}
%
Finally, we formulate a Picard iteration by defining $\mathcal{G}(\mathbf{u}_{i+1}) :=
\mathcal{H}(\mathbf{u}_{i+1}) + \mathbf{u}_{i+1}$, and noting that
$\mathbf{G}(\mathbf{u}_{i+1}) = \mathbf{u}_{i+1}$ if and only if
$\mathcal{F}(\mathbf{u}_{i+1}) = \mathbf{0}$. Fixed-point iterations are then
given by 
%
\begin{align*}
\mathbf{u}_{i+1}^{k+1} & = \mathcal{G}(\mathbf{u}_{i+1}^k) \\
& = (M - \delta t K(\mathbf{u}_{i+1}^k))^{-1}(M\mathbf{u}_i + \mathbf{f}), \\
\Longleftrightarrow\hspace{5ex}
(M - \delta t K(\mathbf{u}_{i+1}^k))\mathbf{u}_{i+1}^k& = M\mathbf{u}_i + \mathbf{f}.\
\end{align*}
%

In fact, this is exactly a Newton-like iteration approximating the Jacobian of
$\mathcal{F}(\mathbf{u}_{i+1})$ evaluated at $\mathbf{u}_{i+1}^k$ as
$J[\mathbf{u}_{i+1}^k] \approx (M - \delta t K(\mathbf{u}_{i+1}^k))$. We use
this observation to extend the same principle to IRK methods, approximating
the Jacobian for the stage equations \eqref{eq:stages} evaluated at the previous
stage vectors as
%
\begin{align*}
&J[\mathbf{k}_1^{k},...,\mathbf{k}_s^k] \approx \\
&\begin{bmatrix} M \\ & \ddots & \\ && M \end{bmatrix} - \delta t
	\begin{bmatrix} a_{11}K\left(\mathbf{u}_n + \delta t\sum_{j=1}^s
	a_{1j}\mathbf{k}_j^k\right) & ... & a_{1s}K\left(\mathbf{u}_n + \delta t\sum_{j=1}^s
	a_{ij}\mathbf{k}_j^k\right) \\
	\vdots & \ddots & \vdots \\
	a_{s1}K\left(\mathbf{u}_n + \delta t\sum_{j=1}^s
	a_{sj}\mathbf{k}_j^k\right) & ... & a_{ss}K\left(\mathbf{u}_n + \delta t\sum_{j=1}^s
	a_{sj}\mathbf{k}_j^k\right)
	\end{bmatrix}.
\end{align*}
%
Letting $\mathcal{L}_i := K\left(\mathbf{u}_n + \delta t\sum_{j=1}^s
a_{ij}\mathbf{k}_j^k\right)$ denote the approximate jacobian for the
$i$th row, we can proceed exactly as discussed in \Cref{sec:nonlinear}.

%
\begin{remark}[Picard as a Newton-like method]
Note, it is important to express Picard and fixed-point iterations in the 
form of a Newton-like method to use the preconditioning techniques developed
in this paper! By expressing it as a Newton-like method, we are approximating
the Jacobian and will still converge to the solution of the nonlinear functional
$\mathcal{F} = 0$ when adding levels of approximation as in \Cref{sec:nonlinear:gen}.
A traditional Picard iteration relies on converging exactly to the fixed-point
$\mathcal{G}(\mathbf{k}) = \mathbf{k}$ via iterates $\mathbf{k}^{k+1} =
\mathcal{G}(\mathbf{k}^k)$. Within this iteration there is delicate cancellation
that result from solving directly for $\mathbf{k}^{k+1}$ rather than a correction,
and approximating $\mathcal{G}$ would result in convergence to a different problem.
\end{remark}
%

% ------------------------------------------------------------------------------------- %
% ------------------------------------------------------------------------------------- %
\subsection{Compressible Navier Stokes}\label{sec:numerics_nonlin:ns}





% ------------------------------------------------------------------------------------- %
% ------------------------------------------------------------------------------------- %
% ------------------------------------------------------------------------------------- %
\section{Conclusions}\label{sec:conc}

This paper introduced a theoretical and algorithmic framework for the fast, parallel
solution of fully implicit Runge-Kutta methods in numerical PDEs (without algebraic
constraints). A field-of-values analysis is derived to guarantee rapid Krylov
convergence...\todo{finish}


% % ------------------------------------------------------------------------------------- %
% % ------------------------------------------------------------------------------------- %
% % ------------------------------------------------------------------------------------- %
% \section*{Appendix}

% % ------------------------------------------------------------------------------------- %
% % ------------------------------------------------------------------------------------- %
% \subsection{Real Schur decomposition applied to general operators}

% Here we write out $(Q_0^T\otimes I)\boldsymbol{\mathcal{L}}(Q_0\otimes I)$
% in bracket notation, where, e.g., $\{a_1,a_2,a_3\}\mapsto a_1\mathcal{L}_1 + a_2\mathcal{L}_2 + 
% a_3\mathcal{L}_3$. Note in the matrices below, the constants in diagonal brackets always
% sum to one and off-diagonal sum to zero, wherein if $\mathcal{L}_i = \mathcal{L}_j$ for
% all $i,j$, each of the following operators is the identity.

% % ------------------------------------------------------------------------------------- %
% \subsubsection*{Gauss}

% %
% \textbf{2-stage:}
% \begin{align*}
% \begin{bmatrix}
% \{1,0\} & \{0,0\} \\
%  \{0,0\} & \{0,1\} \\
% \end{bmatrix}
% \end{align*}
% %
% \textbf{3,-stage:}
% \begin{align*}
% \begin{bmatrix}
% \{0.002,0.565,0.432\} & \{-0.038,-0.245,0.284\} & \{0.024,-0.430,0.405\}\\
% \{-0.038,-0.245,0.284\} & \{0.705,0.107,0.187\} & \{-0.453,0.187,0.266\}\\
% \{0.024,-0.430,0.405\} & \{-0.453,0.187,0.266\} & \{0.291,0.327,0.380\}\\
% \end{bmatrix}
% \end{align*}
% %
% \textbf{4,-stage:}
% \begin{align*}
% \begin{bmatrix}
% \{0.002,0.014,0.012,0.970\} & \{0.001,-0.011,-0.110,0.120\} & \{-0.016,-0.113,0.010,0.119\} & \{-0.045,0.041,-0.005,0.009\}\\
% \{0.001,-0.011,-0.110,0.120\} & \{0.000,0.008,0.975,0.014\} & \{-0.007,0.085,-0.093,0.014\} & \{-0.021,-0.031,0.051,0.001\}\\
% \{-0.016,-0.113,0.010,0.119\} & \{-0.007,0.085,-0.093,0.014\} & \{0.113,0.863,0.008,0.014\} & \{0.316,-0.312,-0.004,0.001\}\\
% \{-0.045,0.041,-0.005,0.009\} & \{-0.021,-0.031,0.051,0.001\} & \{0.316,-0.312,-0.004,0.001\} & \{0.883,0.113,0.002,0.000\}\\
% \end{bmatrix}
% \end{align*}


% %,-,-,-,-,-,-,-,-,-,-,-,-,-,-,-,-,-,-,-,-,-,-,-,-,-,-,-,-,-,-,-,-,-,-,-,-,-,-,-,-,-,-,-,-,-,-,-,-,-,-,-,-,-,-,-,-,-,-,-,-,-,-,-,-,-,-,-,-,-,-,-,-,-,-,-,-,-,-,-,-,-,-,-,-,-%
% \subsubsection*{RadauIIA}

% %
% \textbf{2,-stage:}
% \begin{align*}
% \begin{bmatrix}
% \{0.985,0.014\} & \{0.121,-0.121\}\\
% \{0.121,-0.121\} & \{0.014,0.985\}\\
% \end{bmatrix}
% \end{align*}
% %
% \textbf{3,-stage:}
% \begin{align*}
% \begin{bmatrix}
% \{0.019,0.052,0.928\} & \{0.006,0.222,-0.229\} & \{0.137,-0.017,-0.119\}\\
% \{0.006,0.222,-0.229\} & \{0.002,0.941,0.056\} & \{0.045,-0.075,0.029\}\\
% \{0.137,-0.017,-0.119\} & \{0.045,-0.075,0.029\} & \{0.978,0.006,0.015\}\\
% \end{bmatrix}
% \end{align*}
% %
% \textbf{4,-stage:}
% \begin{align*}
% \begin{bmatrix}
% \{0.002,0.031,0.103,0.862\} & \{0.006,0.022,-0.301,0.272\} & \{0.052,-0.049,0.025,-0.028\} & \{-0.014,-0.166,-0.029,0.209\}\\
% \{0.006,0.022,-0.301,0.272\} & \{0.015,0.016,0.882,0.085\} & \{0.120,-0.035,-0.075,-0.009\} & \{-0.032,-0.118,0.084,0.066\}\\
% \{0.052,-0.049,0.025,-0.028\} & \{0.120,-0.035,-0.075,-0.009\} & \{0.915,0.077,0.006,0.000\} & \{-0.245,0.260,-0.007,-0.006\}\\
% \{-0.014,-0.166,-0.029,0.209\} & \{-0.032,-0.118,0.084,0.066\} & \{-0.245,0.260,-0.007,-0.006\} & \{0.066,0.874,0.008,0.050\}\\
% \end{bmatrix}
% \end{align*}

% ------------------------------------------------------------------------------- %
\bibliographystyle{siamplain}
\bibliography{refs2.bib}


\end{document}


ADJUGATE FORMS, b^T * A0^{-1} * Adj(Ms)
Let M = A0^{-1}, with entries {m_ij}, b = b[b1,...,bs], and xx = spatial operator L

Stiffly accurate RK (b0^TA0^{-1} = [0,...,0,1])
-----------------------------------------------
s = 2
  -m21,
  m11 - xx

s = 3
  -m22 m31 + m21 m32 + m31 xx, 
  m12 m31 - m11 m32 + m32 xx,
  -m12 m21 + m11 m22 - m11 xx - m22 xx + xx^2

s = 4
  m23 m32 m41 - m22 m33 m41 - m23 m31 m42 + m21 m33 m42 + m22 m31 m43 - 
	 m21 m32 m43 + m22 m41 xx + m33 m41 xx - m21 m42 xx - m31 m43 xx - 
	 m41 xx^2,
  -m13 m32 m41 + m12 m33 m41 + m13 m31 m42 - m11 m33 m42 - 
	 m12 m31 m43 + m11 m32 m43 - m12 m41 xx + m11 m42 xx + m33 m42 xx - 
	 m32 m43 xx - m42 xx^2, 
  m13 m22 m41 - m12 m23 m41 - m13 m21 m42 + m11 m23 m42 + m12 m21 m43 - 
	 m11 m22 m43 - m13 m41 xx - m23 m42 xx + m11 m43 xx + m22 m43 xx - 
	 m43 xx^2,
  -m13 m22 m31 + m12 m23 m31 + m13 m21 m32 - m11 m23 m32 - 
	 m12 m21 m33 + m11 m22 m33 + m12 m21 xx - m11 m22 xx + m13 m31 xx + 
	 m23 m32 xx - m11 m33 xx - m22 m33 xx + m11 xx^2 + m22 xx^2 + 
	 m33 xx^2 - xx^3

s = 5
  m24 m33 m42 m51 - m23 m34 m42 m51 - m24 m32 m43 m51 + 
	  m22 m34 m43 m51 + m23 m32 m44 m51 - m22 m33 m44 m51 - 
	  m24 m33 m41 m52 + m23 m34 m41 m52 + m24 m31 m43 m52 - 
	  m21 m34 m43 m52 - m23 m31 m44 m52 + m21 m33 m44 m52 + 
	  m24 m32 m41 m53 - m22 m34 m41 m53 - m24 m31 m42 m53 + 
	  m21 m34 m42 m53 + m22 m31 m44 m53 - m21 m32 m44 m53 - 
	  m23 m32 m41 m54 + m22 m33 m41 m54 + m23 m31 m42 m54 - 
	  m21 m33 m42 m54 - m22 m31 m43 m54 + m21 m32 m43 m54 - 
	  m23 m32 m51 xx + m22 m33 m51 xx - m24 m42 m51 xx - m34 m43 m51 xx + 
	  m22 m44 m51 xx + m33 m44 m51 xx + m23 m31 m52 xx - m21 m33 m52 xx + 
	  m24 m41 m52 xx - m21 m44 m52 xx - m22 m31 m53 xx + m21 m32 m53 xx + 
	  m34 m41 m53 xx - m31 m44 m53 xx - m22 m41 m54 xx - m33 m41 m54 xx + 
	  m21 m42 m54 xx + m31 m43 m54 xx - m22 m51 xx^2 - m33 m51 xx^2 - 
	  m44 m51 xx^2 + m21 m52 xx^2 + m31 m53 xx^2 + m41 m54 xx^2 + 
	  m51 xx^3,
  -m14 m33 m42 m51 + m13 m34 m42 m51 + m14 m32 m43 m51 - 
	  m12 m34 m43 m51 - m13 m32 m44 m51 + m12 m33 m44 m51 + 
	  m14 m33 m41 m52 - m13 m34 m41 m52 - m14 m31 m43 m52 + 
	  m11 m34 m43 m52 + m13 m31 m44 m52 - m11 m33 m44 m52 - 
	  m14 m32 m41 m53 + m12 m34 m41 m53 + m14 m31 m42 m53 - 
	  m11 m34 m42 m53 - m12 m31 m44 m53 + m11 m32 m44 m53 + 
	  m13 m32 m41 m54 - m12 m33 m41 m54 - m13 m31 m42 m54 + 
	  m11 m33 m42 m54 + m12 m31 m43 m54 - m11 m32 m43 m54 + 
	  m13 m32 m51 xx - m12 m33 m51 xx + m14 m42 m51 xx - m12 m44 m51 xx - 
	  m13 m31 m52 xx + m11 m33 m52 xx - m14 m41 m52 xx - m34 m43 m52 xx + 
	  m11 m44 m52 xx + m33 m44 m52 xx + m12 m31 m53 xx - m11 m32 m53 xx + 
	  m34 m42 m53 xx - m32 m44 m53 xx + m12 m41 m54 xx - m11 m42 m54 xx - 
	  m33 m42 m54 xx + m32 m43 m54 xx + m12 m51 xx^2 - m11 m52 xx^2 - 
	  m33 m52 xx^2 - m44 m52 xx^2 + m32 m53 xx^2 + m42 m54 xx^2 + 
	  m52 xx^3, 
  m14 m23 m42 m51 - m13 m24 m42 m51 - m14 m22 m43 m51 + 
	  m12 m24 m43 m51 + m13 m22 m44 m51 - m12 m23 m44 m51 - 
	  m14 m23 m41 m52 + m13 m24 m41 m52 + m14 m21 m43 m52 - 
	  m11 m24 m43 m52 - m13 m21 m44 m52 + m11 m23 m44 m52 + 
	  m14 m22 m41 m53 - m12 m24 m41 m53 - m14 m21 m42 m53 + 
	  m11 m24 m42 m53 + m12 m21 m44 m53 - m11 m22 m44 m53 - 
	  m13 m22 m41 m54 + m12 m23 m41 m54 + m13 m21 m42 m54 - 
	  m11 m23 m42 m54 - m12 m21 m43 m54 + m11 m22 m43 m54 - 
	  m13 m22 m51 xx + m12 m23 m51 xx + m14 m43 m51 xx - m13 m44 m51 xx + 
	  m13 m21 m52 xx - m11 m23 m52 xx + m24 m43 m52 xx - m23 m44 m52 xx - 
	  m12 m21 m53 xx + m11 m22 m53 xx - m14 m41 m53 xx - m24 m42 m53 xx + 
	  m11 m44 m53 xx + m22 m44 m53 xx + m13 m41 m54 xx + m23 m42 m54 xx - 
	  m11 m43 m54 xx - m22 m43 m54 xx + m13 m51 xx^2 + m23 m52 xx^2 - 
	  m11 m53 xx^2 - m22 m53 xx^2 - m44 m53 xx^2 + m43 m54 xx^2 + 
	  m53 xx^3,
  -m14 m23 m32 m51 + m13 m24 m32 m51 + m14 m22 m33 m51 - 
	  m12 m24 m33 m51 - m13 m22 m34 m51 + m12 m23 m34 m51 + 
	  m14 m23 m31 m52 - m13 m24 m31 m52 - m14 m21 m33 m52 + 
	  m11 m24 m33 m52 + m13 m21 m34 m52 - m11 m23 m34 m52 - 
	  m14 m22 m31 m53 + m12 m24 m31 m53 + m14 m21 m32 m53 - 
	  m11 m24 m32 m53 - m12 m21 m34 m53 + m11 m22 m34 m53 + 
	  m13 m22 m31 m54 - m12 m23 m31 m54 - m13 m21 m32 m54 + 
	  m11 m23 m32 m54 + m12 m21 m33 m54 - m11 m22 m33 m54 - 
	  m14 m22 m51 xx + m12 m24 m51 xx - m14 m33 m51 xx + m13 m34 m51 xx + 
	  m14 m21 m52 xx - m11 m24 m52 xx - m24 m33 m52 xx + m23 m34 m52 xx + 
	  m14 m31 m53 xx + m24 m32 m53 xx - m11 m34 m53 xx - m22 m34 m53 xx - 
	  m12 m21 m54 xx + m11 m22 m54 xx - m13 m31 m54 xx - m23 m32 m54 xx + 
	  m11 m33 m54 xx + m22 m33 m54 xx + m14 m51 xx^2 + m24 m52 xx^2 + 
	  m34 m53 xx^2 - m11 m54 xx^2 - m22 m54 xx^2 - m33 m54 xx^2 + 
	  m54 xx^3, 
  m14 m23 m32 m41 - m13 m24 m32 m41 - m14 m22 m33 m41 + 
	  m12 m24 m33 m41 + m13 m22 m34 m41 - m12 m23 m34 m41 - 
	  m14 m23 m31 m42 + m13 m24 m31 m42 + m14 m21 m33 m42 - 
	  m11 m24 m33 m42 - m13 m21 m34 m42 + m11 m23 m34 m42 + 
	  m14 m22 m31 m43 - m12 m24 m31 m43 - m14 m21 m32 m43 + 
	  m11 m24 m32 m43 + m12 m21 m34 m43 - m11 m22 m34 m43 - 
	  m13 m22 m31 m44 + m12 m23 m31 m44 + m13 m21 m32 m44 - 
	  m11 m23 m32 m44 - m12 m21 m33 m44 + m11 m22 m33 m44 + 
	  m13 m22 m31 xx - m12 m23 m31 xx - m13 m21 m32 xx + m11 m23 m32 xx + 
	  m12 m21 m33 xx - m11 m22 m33 xx + m14 m22 m41 xx - m12 m24 m41 xx + 
	  m14 m33 m41 xx - m13 m34 m41 xx - m14 m21 m42 xx + m11 m24 m42 xx + 
	  m24 m33 m42 xx - m23 m34 m42 xx - m14 m31 m43 xx - m24 m32 m43 xx + 
	  m11 m34 m43 xx + m22 m34 m43 xx + m12 m21 m44 xx - m11 m22 m44 xx + 
	  m13 m31 m44 xx + m23 m32 m44 xx - m11 m33 m44 xx - m22 m33 m44 xx - 
	  m12 m21 xx^2 + m11 m22 xx^2 - m13 m31 xx^2 - m23 m32 xx^2 + 
	  m11 m33 xx^2 + m22 m33 xx^2 - m14 m41 xx^2 - m24 m42 xx^2 - 
	  m34 m43 xx^2 + m11 m44 xx^2 + m22 m44 xx^2 + m33 m44 xx^2 - 
	  m11 xx^3 - m22 xx^3 - m33 xx^3 - m44 xx^3 + xx^4

Full implicit RK
----------------
s = 2
  -b1 m12 m21 + b1 m11 m22 - b1 m11 xx - b2 m21 xx,
  -b2 m12 m21 + b2 m11 m22 - b1 m12 xx - b2 m22 xx

s = 3
  -b1 m13 m22 m31 + b1 m12 m23 m31 + b1 m13 m21 m32 - b1 m11 m23 m32 -
     b1 m12 m21 m33 + b1 m11 m22 m33 + b1 m12 m21 xx - b1 m11 m22 xx + 
     b1 m13 m31 xx - b3 m22 m31 xx + b2 m23 m31 xx + b3 m21 m32 xx - 
     b1 m11 m33 xx - b2 m21 m33 xx + b1 m11 xx^2 + b2 m21 xx^2 + 
     b3 m31 xx^2,
  -b2 m13 m22 m31 + b2 m12 m23 m31 + b2 m13 m21 m32 - 
     b2 m11 m23 m32 - b2 m12 m21 m33 + b2 m11 m22 m33 + b2 m12 m21 xx - 
     b2 m11 m22 xx + b3 m12 m31 xx - b3 m11 m32 xx + b1 m13 m32 xx + 
     b2 m23 m32 xx - b1 m12 m33 xx - b2 m22 m33 xx + b1 m12 xx^2 + 
     b2 m22 xx^2 + b3 m32 xx^2,
  -b3 m13 m22 m31 + b3 m12 m23 m31 + 
     b3 m13 m21 m32 - b3 m11 m23 m32 - b3 m12 m21 m33 + b3 m11 m22 m33 +
     b2 m13 m21 xx - b1 m13 m22 xx - b2 m11 m23 xx + b1 m12 m23 xx + 
     b3 m13 m31 xx + b3 m23 m32 xx - b3 m11 m33 xx - b3 m22 m33 xx + 
     b1 m13 xx^2 + b2 m23 xx^2 + b3 m33 xx^2

s = 4
  b1 m14 m23 m32 m41 - b1 m13 m24 m32 m41 - b1 m14 m22 m33 m41 + 
	  b1 m12 m24 m33 m41 + b1 m13 m22 m34 m41 - b1 m12 m23 m34 m41 - 
	  b1 m14 m23 m31 m42 + b1 m13 m24 m31 m42 + b1 m14 m21 m33 m42 - 
	  b1 m11 m24 m33 m42 - b1 m13 m21 m34 m42 + b1 m11 m23 m34 m42 + 
	  b1 m14 m22 m31 m43 - b1 m12 m24 m31 m43 - b1 m14 m21 m32 m43 + 
	  b1 m11 m24 m32 m43 + b1 m12 m21 m34 m43 - b1 m11 m22 m34 m43 - 
	  b1 m13 m22 m31 m44 + b1 m12 m23 m31 m44 + b1 m13 m21 m32 m44 - 
	  b1 m11 m23 m32 m44 - b1 m12 m21 m33 m44 + b1 m11 m22 m33 m44 + 
	  b1 m13 m22 m31 xx - b1 m12 m23 m31 xx - b1 m13 m21 m32 xx + 
	  b1 m11 m23 m32 xx + b1 m12 m21 m33 xx - b1 m11 m22 m33 xx + 
	  b1 m14 m22 m41 xx - b1 m12 m24 m41 xx + b4 m23 m32 m41 xx - 
	  b3 m24 m32 m41 xx + b1 m14 m33 m41 xx - b4 m22 m33 m41 xx + 
	  b2 m24 m33 m41 xx - b1 m13 m34 m41 xx + b3 m22 m34 m41 xx - 
	  b2 m23 m34 m41 xx - b1 m14 m21 m42 xx + b1 m11 m24 m42 xx - 
	  b4 m23 m31 m42 xx + b3 m24 m31 m42 xx + b4 m21 m33 m42 xx - 
	  b3 m21 m34 m42 xx - b1 m14 m31 m43 xx + b4 m22 m31 m43 xx - 
	  b2 m24 m31 m43 xx - b4 m21 m32 m43 xx + b1 m11 m34 m43 xx + 
	  b2 m21 m34 m43 xx + b1 m12 m21 m44 xx - b1 m11 m22 m44 xx + 
	  b1 m13 m31 m44 xx - b3 m22 m31 m44 xx + b2 m23 m31 m44 xx + 
	  b3 m21 m32 m44 xx - b1 m11 m33 m44 xx - b2 m21 m33 m44 xx - 
	  b1 m12 m21 xx^2 + b1 m11 m22 xx^2 - b1 m13 m31 xx^2 + 
	  b3 m22 m31 xx^2 - b2 m23 m31 xx^2 - b3 m21 m32 xx^2 + 
	  b1 m11 m33 xx^2 + b2 m21 m33 xx^2 - b1 m14 m41 xx^2 + 
	  b4 m22 m41 xx^2 - b2 m24 m41 xx^2 + b4 m33 m41 xx^2 - 
	  b3 m34 m41 xx^2 - b4 m21 m42 xx^2 - b4 m31 m43 xx^2 + 
	  b1 m11 m44 xx^2 + b2 m21 m44 xx^2 + b3 m31 m44 xx^2 - b1 m11 xx^3 - 
	  b2 m21 xx^3 - b3 m31 xx^3 - b4 m41 xx^3, 
   b2 m14 m23 m32 m41 - b2 m13 m24 m32 m41 - b2 m14 m22 m33 m41 + 
	  b2 m12 m24 m33 m41 + b2 m13 m22 m34 m41 - b2 m12 m23 m34 m41 - 
	  b2 m14 m23 m31 m42 + b2 m13 m24 m31 m42 + b2 m14 m21 m33 m42 - 
	  b2 m11 m24 m33 m42 - b2 m13 m21 m34 m42 + b2 m11 m23 m34 m42 + 
	  b2 m14 m22 m31 m43 - b2 m12 m24 m31 m43 - b2 m14 m21 m32 m43 + 
	  b2 m11 m24 m32 m43 + b2 m12 m21 m34 m43 - b2 m11 m22 m34 m43 - 
	  b2 m13 m22 m31 m44 + b2 m12 m23 m31 m44 + b2 m13 m21 m32 m44 - 
	  b2 m11 m23 m32 m44 - b2 m12 m21 m33 m44 + b2 m11 m22 m33 m44 + 
	  b2 m13 m22 m31 xx - b2 m12 m23 m31 xx - b2 m13 m21 m32 xx + 
	  b2 m11 m23 m32 xx + b2 m12 m21 m33 xx - b2 m11 m22 m33 xx + 
	  b2 m14 m22 m41 xx - b2 m12 m24 m41 xx - b4 m13 m32 m41 xx + 
	  b3 m14 m32 m41 xx + b4 m12 m33 m41 xx - b3 m12 m34 m41 xx - 
	  b2 m14 m21 m42 xx + b2 m11 m24 m42 xx + b4 m13 m31 m42 xx - 
	  b3 m14 m31 m42 xx - b4 m11 m33 m42 xx + b1 m14 m33 m42 xx + 
	  b2 m24 m33 m42 xx + b3 m11 m34 m42 xx - b1 m13 m34 m42 xx - 
	  b2 m23 m34 m42 xx - b4 m12 m31 m43 xx + b4 m11 m32 m43 xx - 
	  b1 m14 m32 m43 xx - b2 m24 m32 m43 xx + b1 m12 m34 m43 xx + 
	  b2 m22 m34 m43 xx + b2 m12 m21 m44 xx - b2 m11 m22 m44 xx + 
	  b3 m12 m31 m44 xx - b3 m11 m32 m44 xx + b1 m13 m32 m44 xx + 
	  b2 m23 m32 m44 xx - b1 m12 m33 m44 xx - b2 m22 m33 m44 xx - 
	  b2 m12 m21 xx^2 + b2 m11 m22 xx^2 - b3 m12 m31 xx^2 + 
	  b3 m11 m32 xx^2 - b1 m13 m32 xx^2 - b2 m23 m32 xx^2 + 
	  b1 m12 m33 xx^2 + b2 m22 m33 xx^2 - b4 m12 m41 xx^2 + 
	  b4 m11 m42 xx^2 - b1 m14 m42 xx^2 - b2 m24 m42 xx^2 + 
	  b4 m33 m42 xx^2 - b3 m34 m42 xx^2 - b4 m32 m43 xx^2 + 
	  b1 m12 m44 xx^2 + b2 m22 m44 xx^2 + b3 m32 m44 xx^2 - b1 m12 xx^3 - 
	  b2 m22 xx^3 - b3 m32 xx^3 - b4 m42 xx^3, 
   b3 m14 m23 m32 m41 - b3 m13 m24 m32 m41 - b3 m14 m22 m33 m41 + 
	  b3 m12 m24 m33 m41 + b3 m13 m22 m34 m41 - b3 m12 m23 m34 m41 - 
	  b3 m14 m23 m31 m42 + b3 m13 m24 m31 m42 + b3 m14 m21 m33 m42 - 
	  b3 m11 m24 m33 m42 - b3 m13 m21 m34 m42 + b3 m11 m23 m34 m42 + 
	  b3 m14 m22 m31 m43 - b3 m12 m24 m31 m43 - b3 m14 m21 m32 m43 + 
	  b3 m11 m24 m32 m43 + b3 m12 m21 m34 m43 - b3 m11 m22 m34 m43 - 
	  b3 m13 m22 m31 m44 + b3 m12 m23 m31 m44 + b3 m13 m21 m32 m44 - 
	  b3 m11 m23 m32 m44 - b3 m12 m21 m33 m44 + b3 m11 m22 m33 m44 + 
	  b3 m13 m22 m31 xx - b3 m12 m23 m31 xx - b3 m13 m21 m32 xx + 
	  b3 m11 m23 m32 xx + b3 m12 m21 m33 xx - b3 m11 m22 m33 xx + 
	  b4 m13 m22 m41 xx - b4 m12 m23 m41 xx + b2 m14 m23 m41 xx - 
	  b2 m13 m24 m41 xx + b3 m14 m33 m41 xx - b3 m13 m34 m41 xx - 
	  b4 m13 m21 m42 xx + b4 m11 m23 m42 xx - b1 m14 m23 m42 xx + 
	  b1 m13 m24 m42 xx + b3 m24 m33 m42 xx - b3 m23 m34 m42 xx + 
	  b4 m12 m21 m43 xx - b2 m14 m21 m43 xx - b4 m11 m22 m43 xx + 
	  b1 m14 m22 m43 xx + b2 m11 m24 m43 xx - b1 m12 m24 m43 xx - 
	  b3 m14 m31 m43 xx - b3 m24 m32 m43 xx + b3 m11 m34 m43 xx + 
	  b3 m22 m34 m43 xx + b2 m13 m21 m44 xx - b1 m13 m22 m44 xx - 
	  b2 m11 m23 m44 xx + b1 m12 m23 m44 xx + b3 m13 m31 m44 xx + 
	  b3 m23 m32 m44 xx - b3 m11 m33 m44 xx - b3 m22 m33 m44 xx - 
	  b2 m13 m21 xx^2 + b1 m13 m22 xx^2 + b2 m11 m23 xx^2 - 
	  b1 m12 m23 xx^2 - b3 m13 m31 xx^2 - b3 m23 m32 xx^2 + 
	  b3 m11 m33 xx^2 + b3 m22 m33 xx^2 - b4 m13 m41 xx^2 - 
	  b4 m23 m42 xx^2 + b4 m11 m43 xx^2 - b1 m14 m43 xx^2 + 
	  b4 m22 m43 xx^2 - b2 m24 m43 xx^2 - b3 m34 m43 xx^2 + 
	  b1 m13 m44 xx^2 + b2 m23 m44 xx^2 + b3 m33 m44 xx^2 - b1 m13 xx^3 - 
	  b2 m23 xx^3 - b3 m33 xx^3 - b4 m43 xx^3, 
   b4 m14 m23 m32 m41 - b4 m13 m24 m32 m41 - b4 m14 m22 m33 m41 + 
	  b4 m12 m24 m33 m41 + b4 m13 m22 m34 m41 - b4 m12 m23 m34 m41 - 
	  b4 m14 m23 m31 m42 + b4 m13 m24 m31 m42 + b4 m14 m21 m33 m42 - 
	  b4 m11 m24 m33 m42 - b4 m13 m21 m34 m42 + b4 m11 m23 m34 m42 + 
	  b4 m14 m22 m31 m43 - b4 m12 m24 m31 m43 - b4 m14 m21 m32 m43 + 
	  b4 m11 m24 m32 m43 + b4 m12 m21 m34 m43 - b4 m11 m22 m34 m43 - 
	  b4 m13 m22 m31 m44 + b4 m12 m23 m31 m44 + b4 m13 m21 m32 m44 - 
	  b4 m11 m23 m32 m44 - b4 m12 m21 m33 m44 + b4 m11 m22 m33 m44 + 
	  b3 m14 m22 m31 xx - b2 m14 m23 m31 xx - b3 m12 m24 m31 xx + 
	  b2 m13 m24 m31 xx - b3 m14 m21 m32 xx + b1 m14 m23 m32 xx + 
	  b3 m11 m24 m32 xx - b1 m13 m24 m32 xx + b2 m14 m21 m33 xx - 
	  b1 m14 m22 m33 xx - b2 m11 m24 m33 xx + b1 m12 m24 m33 xx + 
	  b3 m12 m21 m34 xx - b2 m13 m21 m34 xx - b3 m11 m22 m34 xx + 
	  b1 m13 m22 m34 xx + b2 m11 m23 m34 xx - b1 m12 m23 m34 xx + 
	  b4 m14 m22 m41 xx - b4 m12 m24 m41 xx + b4 m14 m33 m41 xx - 
	  b4 m13 m34 m41 xx - b4 m14 m21 m42 xx + b4 m11 m24 m42 xx + 
	  b4 m24 m33 m42 xx - b4 m23 m34 m42 xx - b4 m14 m31 m43 xx - 
	  b4 m24 m32 m43 xx + b4 m11 m34 m43 xx + b4 m22 m34 m43 xx + 
	  b4 m12 m21 m44 xx - b4 m11 m22 m44 xx + b4 m13 m31 m44 xx + 
	  b4 m23 m32 m44 xx - b4 m11 m33 m44 xx - b4 m22 m33 m44 xx - 
	  b2 m14 m21 xx^2 + b1 m14 m22 xx^2 + b2 m11 m24 xx^2 - 
	  b1 m12 m24 xx^2 - b3 m14 m31 xx^2 - b3 m24 m32 xx^2 + 
	  b1 m14 m33 xx^2 + b2 m24 m33 xx^2 + b3 m11 m34 xx^2 - 
	  b1 m13 m34 xx^2 + b3 m22 m34 xx^2 - b2 m23 m34 xx^2 - 
	  b4 m14 m41 xx^2 - b4 m24 m42 xx^2 - b4 m34 m43 xx^2 + 
	  b4 m11 m44 xx^2 + b4 m22 m44 xx^2 + b4 m33 m44 xx^2 - b1 m14 xx^3 - 
	  b2 m24 xx^3 - b3 m34 xx^3 - b4 m44 xx^3


Computing A0^{-1} = det(A0)^{-1}Adj(A0)
---------------------------------------
From Wikpedia, The adjugate of A is the n×n matrix whose (i,j) entry is the (j,i)
cofactor of A, (-1)^{i+j} * M_ij, where M_ij is the determinant of the principle
minor of A that comes form deleting rows i and j. Moreover, using the Laplace
formula, computing these minors also yields det(A):
	https://en.wikipedia.org/wiki/Determinant#Laplace's_formula_and_the_adjugate_matrix
I think we should make functions that take an MFEM dense matrix and compute determinants
for a given set of rows and columns, e.g., write the following function for 2,3, and 4
sets of rows/columns:

getMinorDet(DenseMatrix A, int row1, int row2, int col1, int col2)
{
	if (row2 >= A.Height() || col2 >0 A.Width()) {
		error
	}
	return A[row1,col1]*A[row2,col2] - A[row1,col2]*A[row2,col1];
}

Using Laplace formula and adjugate/determinant formula for inverse, these would provide
algebraic inverses for RK tableauxs up to s = 5 with minimal code (could go higher, just
need to add more determinants; 5 stages is probably plenty to start).

Det of 3x3:
-m13 m22 m31 + m12 m23 m31 + m13 m21 m32 - m11 m23 m32 - m12 m21 m33 +
  m11 m22 m33

Det of 4x4:
m14 m23 m32 m41 - m13 m24 m32 m41 - m14 m22 m33 m41 + 
 m12 m24 m33 m41 + m13 m22 m34 m41 - m12 m23 m34 m41 - 
 m14 m23 m31 m42 + m13 m24 m31 m42 + m14 m21 m33 m42 - 
 m11 m24 m33 m42 - m13 m21 m34 m42 + m11 m23 m34 m42 + 
 m14 m22 m31 m43 - m12 m24 m31 m43 - m14 m21 m32 m43 + 
 m11 m24 m32 m43 + m12 m21 m34 m43 - m11 m22 m34 m43 - 
 m13 m22 m31 m44 + m12 m23 m31 m44 + m13 m21 m32 m44 - 
 m11 m23 m32 m44 - m12 m21 m33 m44 + m11 m22 m33 m44

